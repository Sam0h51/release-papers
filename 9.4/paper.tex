\documentclass{ansarticle-preprint}
%\usepackage{ucs}
\usepackage[utf8]{inputenc}
\usepackage{amsmath}
%\usepackage{cite}
\usepackage{anslistings}
\usepackage{multicol}
\usepackage{pdfsync}

\usepackage{pgfplots}
\usepackage{pgfplotstable}

\usepackage{fontenc}
\usepackage{graphicx}
\usepackage{xspace}

\usepackage{siunitx}

\usepackage{floatflt}

\usepackage{multirow}

%\renewcommand{\baselinestretch}{2.0}

\usepackage[normalem]{ulem}

\usepackage{todonotes}

\pgfplotsset{compat=1.9}
\definecolor{gnuplot@lightblue}{RGB}{87,181,232}
\definecolor{gnuplot@green}{RGB}{0,158,115}
\definecolor{gnuplot@purple}{RGB}{148,0,212}

\newcommand{\specialword}[1]{\texttt{#1}}
\newcommand{\dealii}{{\specialword{deal.II}}\xspace}
\newcommand{\pfrst}{{\specialword{p4est}}\xspace}
\newcommand{\trilinos}{{\specialword{Trilinos}}\xspace}
\newcommand{\aspect}{\specialword{Aspect}\xspace}
\newcommand{\petsc}{\specialword{PETSc}\xspace}
\newcommand{\cmake}{{\specialword{CMake}}\xspace}
\newcommand{\candi}{{\specialword{candi}}\xspace}



\usetikzlibrary{shapes.misc}
\tikzset{cross/.style={cross out, draw=black, minimum size=2*(#1-\pgflinewidth), inner sep=0pt, outer sep=0pt},
%default radius will be 1pt.
cross/.default={2pt}}

%
% Author list -- please add yourself in both places below (in
%                alphabetical order) if you think that your
%                contributions to the last release warrant this
%

\hypersetup{
  pdfauthor={
    Daniel Arndt,
    Wolfgang Bangerth,
    Marc Fehling,
    Timo Heister,
    Luca Heltai,
    Martin Kronbichler,
    Matthias Maier,
    Peter Munch,
    Jean-Paul Pelteret,
    Bruno Turcksin,
    David Wells
  },
  pdftitle={The deal.II Library, Version 9.4, 2022},
}

\title{The \dealii{} Library, Version 9.4}

 \author[1*]{Daniel Arndt}
 \affil[1]{Scalable Algorithms and Coupled Physics Group,
   Computational Sciences and Engineering Division,
   Oak Ridge National Laboratory, 1 Bethel Valley Rd.,
   TN 37831, USA.
   \texttt{arndtd/turcksinbr@ornl.gov}}

 \author[2,3]{Wolfgang~Bangerth}
 \affil[2]{Department of Mathematics, Colorado State University, Fort
   Collins, CO 80523-1874, USA.
   \texttt{bangerth/marc.fehling@colostate.edu}}
 \affil[3]{Department of Geosciences, Colorado State University, Fort
   Collins, CO 80523, USA.}

\author[2]{Marc~Fehling}

\author[6]{Timo~Heister}
 \affil[6]{School of Mathematical and Statistical Sciences,
   Clemson University,
   Clemson, SC, 29634, USA
   {\texttt{heister/jiaqi2@clemson.edu}}}

\author[7]{Luca~Heltai}
\affil[7]{SISSA,
   International School for Advanced Studies,
   Via Bonomea 265,
   34136, Trieste, Italy.
   {\texttt{luca.heltai@sissa.it}}}

 \author[9,10]{Martin~Kronbichler}
 \affil[9]{Institute for Computational Mechanics,
   Technical University of Munich,
   Boltzmannstr.~15, 85748 Garching, Germany.
   {\texttt{kronbichler/munch/proell@lnm.mw.tum.de}}}
 \affil[10]{Department of Information Technology,
   Uppsala University,
   Box 337, 751\,05 Uppsala, Sweden.
   {\texttt{martin.kronbichler@it.uu.se}}}

\author[11]{Matthias~Maier}
\affil[11]{Department of Mathematics,
  Texas A\&M University,
  3368 TAMU,
  College Station, TX 77845, USA.
  {\texttt{maier@math.tamu.edu}}}

\author[9,12]{Peter Munch}
 \affil[12]{Institute of Material Systems Modeling,
 Helmholtz-Zentrum Hereon,
 Max-Planck-Str. 1, 21502 Geesthacht, Germany.
   {\texttt{peter.muench@hereon.de}}}


\author[13]{Jean-Paul~Pelteret}
\affil[13]{Independent researcher.
{\texttt{jppelteret@gmail.com}}}

\author[1*]{Bruno~Turcksin}

\author[15]{David Wells}
\affil[15]{Department of Mathematics, University of North Carolina,
  Chapel Hill, NC 27516, USA.
  {\texttt{drwells@email.unc.edu}}}

\renewcommand{\labelitemi}{--}


\begin{document}
\maketitle

\footnotetext{%
  $^\ast$ This manuscript has been authored by UT-Battelle, LLC under Contract No.
  DE-AC05-00OR22725 with the U.S. Department of Energy.
  %The United States
  %Government retains and the publisher, by accepting the article for
  %publication, acknowledges that the United States Government retains a
  %non-exclusive, paid-up, irrevocable, worldwide license to publish or reproduce
  %the published form of this manuscript, or allow others to do so, for United
  %States Government purposes. The Department of Energy will provide public
  %access to these results of federally sponsored research in accordance with the
  %DOE Public Access Plan (http://energy.gov/downloads/doe-public-access-plan).
}


\begin{abstract}
  This paper provides an overview of the new features of the finite element
  library \dealii, version 9.3.
\end{abstract}



%%%%%%%%%%%%%%%%%%%%%%%%%%%%%%%%%%%%%%%%%%%%%%%%%%%%%%%%%%%%%%%%%%%%%%%%%%%%%%%%
%%%%%%%%%%%%%%%%%%%%%%%%%%%%%%%%%%%%%%%%%%%%%%%%%%%%%%%%%%%%%%%%%%%%%%%%%%%%%%%%
%%%%%%%%%%%%%%%%%%%%%%%%%%%%%%%%%%%%%%%%%%%%%%%%%%%%%%%%%%%%%%%%%%%%%%%%%%%%%%%%
\section{Overview}

\dealii{} version 9.4.0 was released June TODO, 2022.
This paper provides an
overview of the new features of this release and serves as a citable
reference for the \dealii{} software library version 9.4. \dealii{} is an
object-oriented finite element library used around the world in the
development of finite element solvers. It is available for free under the
GNU Lesser General Public License (LGPL). Downloads are available at
\url{https://www.dealii.org/} and \url{https://github.com/dealii/dealii}.

The major changes of this release are:
%
\begin{itemize}
  \item TODO;
  \item TODO new tutorial programs and a new code gallery program (see Section~\ref{subsec:steps}).
\end{itemize}
%
In addition, we discuss the \candi{} installation program in Section~\ref{subsec:candi}.

While all of these major changes are discussed in detail in
Section~\ref{sec:major}, there
are a number of other noteworthy changes in the current \dealii{} release
that we briefly outline in the remainder of this section:
%
\begin{itemize}
  \item TODO.
\end{itemize}
%
The changelog lists more than 200 other features and bugfixes.




%%%%%%%%%%%%%%%%%%%%%%%%%%%%%%%%%%%%%%%%%%%%%%%%%%%%%%%%%%%%%%%%%%%%%%%%%%%%%%%%
%%%%%%%%%%%%%%%%%%%%%%%%%%%%%%%%%%%%%%%%%%%%%%%%%%%%%%%%%%%%%%%%%%%%%%%%%%%%%%%%
%%%%%%%%%%%%%%%%%%%%%%%%%%%%%%%%%%%%%%%%%%%%%%%%%%%%%%%%%%%%%%%%%%%%%%%%%%%%%%%%
\section{Major changes to the library}
\label{sec:major}

This release of \dealii{} contains a number of large and significant changes
that will be discussed in this section.
It of course also contains a
vast number of smaller changes and added functionality; the details of these
can be found
\href{https://dealii.org/developer/doxygen/deal.II/changes_between_9_3_0_and_9_4_0.html}
{in the file that lists all changes for this release}; see \cite{changes94}.

%\newpage

%%%%%%%%%%%%%%%%%%%%%%%%%%%%%%%%%%%%%%%%%%%%%%%%%%%%%%%%%%%%%%%%%%%%%%%%%%%%%%%%
\subsection{Topic 1}



%%%%%%%%%%%%%%%%%%%%%%%%%%%%%%%%%%%%%%%%%%%%%%%%%%%%%%%%%%%%%%%%%%%%%%%%%%%%%%%%
\subsection{Topic 2}



%%%%%%%%%%%%%%%%%%%%%%%%%%%%%%%%%%%%%%%%%%%%%%%%%%%%%%%%%%%%%%%%%%%%%%%%%%%%%%%%
\subsection{Topic 3}



%%%%%%%%%%%%%%%%%%%%%%%%%%%%%%%%%%%%%%%%%%%%%%%%%%%%%%%%%%%%%%%%%%%%%%%%%%%%%%%%
\subsection{Topic 4}



%%%%%%%%%%%%%%%%%%%%%%%%%%%%%%%%%%%%%%%%%%%%%%%%%%%%%%%%%%%%%%%%%%%%%%%%%%%%%%%%
\subsection{Topic 5}



%%%%%%%%%%%%%%%%%%%%%%%%%%%%%%%%%%%%%%%%%%%%%%%%%%%%%%%%%%%%%%%%%%%%%%%%%%%%%%%%
\subsection{Topic 6}



%%%%%%%%%%%%%%%%%%%%%%%%%%%%%%%%%%%%%%%%%%%%%%%%%%%%%%%%%%%%%%%%%%%%%%%%%%%%%%%%
\subsection{Topic 7}



%%%%%%%%%%%%%%%%%%%%%%%%%%%%%%%%%%%%%%%%%%%%%%%%%%%%%%%%%%%%%%%%%%%%%%%%%%%%%%%%
\subsection{New and improved tutorials and code gallery programs}
\label{subsec:steps}

Many of the \dealii{} tutorial programs were revised in a variety of
ways as part of this release. In addition, there are a number of new tutorial programs:
\begin{itemize}
\item \texttt{TODO} TODO.


\end{itemize}

There is also a new program in the code gallery (a collection of
user-contributed programs that often solve more complicated problems
than tutorial programs, and intended as starting points for further
research rather than as teaching tools):
\begin{itemize}
\item ``\texttt{TODO}'' TODO.
\end{itemize}



%%%%%%%%%%%%%%%%%%%%%%%%%%%%%%%%%%%%%%%%%%%%%%%%%%%%%%%%%%%%%%%%%%%%%%%%%%%%%%%%
\subsection{Incompatible changes}\label{subsec:deprecated}

The 9.4 release includes
\href{https://dealii.org/developer/doxygen/deal.II/changes_between_9_3_0_and_9_4_0.html}
{around TODO incompatible changes}; see \cite{changes94}. The majority of these changes
should not be visible to typical user codes; some remove previously
deprecated classes and functions; and the majority change internal
interfaces that are not usually used in external
applications. That said, the following are worth mentioning since they
may have been more widely used:
\begin{itemize}
  \item TODO
  \item TODO

\end{itemize}



%%%%%%%%%%%%%%%%%%%%%%%%%%%%%%%%%%%%%%%%%%%%%%%%%%%%%%%%%%%%%%%%%%%%%%%%%%%%%%%%
%%%%%%%%%%%%%%%%%%%%%%%%%%%%%%%%%%%%%%%%%%%%%%%%%%%%%%%%%%%%%%%%%%%%%%%%%%%%%%%%
%%%%%%%%%%%%%%%%%%%%%%%%%%%%%%%%%%%%%%%%%%%%%%%%%%%%%%%%%%%%%%%%%%%%%%%%%%%%%%%%
\section{How to cite \dealii}\label{sec:cite}

In order to justify the work the developers of \dealii{} put into this
software, we ask that papers using the library reference one of the
\dealii{} papers. This helps us justify the effort we put into this library.

There are various ways to reference \dealii{}. To acknowledge the use of
the current version of the library, \textbf{please reference the present
  document}. For up to date information and a bibtex entry
see
\begin{center}
  \url{https://www.dealii.org/publications.html}
\end{center}

The original \dealii{} paper containing an overview of its
architecture is \cite{BangerthHartmannKanschat2007}, and a more recent
publication documenting \dealii{}'s design decisions is available as \cite{dealII2020design}. If you rely on
specific features of the library, please consider citing any of the
following:
\begin{multicols}{2}
  \vspace*{-36pt}
  \begin{itemize}
    \item For geometric multigrid: \cite{Kanschat2004,JanssenKanschat2011,ClevengerHeisterKanschatKronbichler2019};
    \item For distributed parallel computing: \cite{BangerthBursteddeHeisterKronbichler11};
    \item For $hp$-adaptivity: \cite{BangerthKayserHerold2007};
    \item For partition-of-unity (PUM) and finite element enrichment methods:
           \cite{Davydov2016};
    \item For matrix-free and fast assembly techniques:
          \cite{KronbichlerKormann2012,KronbichlerKormann2019};
    \item For computations on lower-dimensional manifolds:
          \cite{DeSimoneHeltaiManigrasso2009};
    \item For curved geometry representations and manifolds:
          \cite{HeltaiBangerthKronbichlerMola2019};
    \item For integration with CAD files and tools:
          \cite{HeltaiMola2015};
    \item For boundary element computations:
          \cite{GiulianiMolaHeltai-2018-a};
    \item For the \texttt{LinearOperator} and
      \texttt{Packaged\-Operation} facilities:
          \cite{MaierBardelloniHeltai-2016-a,MaierBardelloniHeltai-2016-b};
    \item For uses of the \texttt{WorkStream} interface:
          \cite{TKB16};
    \item For uses of the \texttt{ParameterAcceptor} concept, the
          \texttt{MeshWorker::ScratchData} base class, and the
          \texttt{ParsedConvergenceTable} class:
          \cite{SartoriGiulianiBardelloni-2018-a};
    \item For uses of the particle functionality in \dealii{}:
          \cite{GLHPB18}.
          \vfill\null
  \end{itemize}
\end{multicols}

\dealii{} can interface with many other libraries:
\begin{multicols}{3}
  \begin{itemize}
    \item ADOL-C \cite{Griewank1996a,adol-c}
    \item ArborX \cite{lebrun2020arborx}
    \item ARPACK \cite{arpack}
    \item Assimp \cite{assimp}
    \item BLAS and LAPACK \cite{lapack}
    \item cuSOLVER \cite{cusolver}
    \item cuSPARSE \cite{cusparse}
    \item Gmsh \cite{geuzaine2009gmsh}
    \item GSL \cite{gsl2016}
    \item Ginkgo \cite{ginkgo-web-page}
    \item HDF5 \cite{hdf5}
    \item METIS \cite{karypis1998fast}
    \item MUMPS \cite{ADE00,MUMPS:1,MUMPS:2,mumps-web-page}
    \item muparser \cite{muparser-web-page}
    \item OpenCASCADE \cite{opencascade-web-page}
    \item p4est \cite{p4est}
    \item PETSc \cite{petsc-user-ref,petsc-web-page}
    \item ROL \cite{ridzal2014rapid}
    \item ScaLAPACK \cite{slug}
    \item SLEPc \cite{Hernandez:2005:SSF}
    \item SUNDIALS \cite{sundials}
    \item SymEngine \cite{symengine-web-page}
    \item TBB \cite{Rei07}
    \item Trilinos \cite{trilinos,trilinos-web-page}
    \item UMFPACK \cite{umfpack}
  \end{itemize}
\end{multicols}
Please consider citing the appropriate references if you use
interfaces to these libraries. 

The two previous releases of \dealii{} can be cited as
\cite{dealII91,dealII92}.


\section{Acknowledgments}

\dealii{} is a world-wide project with dozens of contributors around the
globe. Other than the authors of this paper, the following people
contributed code to this release:\\
%
% Uwe Koecher doesn't usually show up in the changelog, but
% we should make sure he's listed.
%
% 9.3: updated 5/08/2021 PM
%      sorted by WB 5/28
TODO.

Their contributions are much appreciated!


\bigskip

\dealii{} and its developers are financially supported through a
variety of funding sources:

D.~Arndt and B.~Turcksin: Research sponsored by the Laboratory Directed Research and
Development Program of Oak Ridge National Laboratory, managed by UT-Battelle,
LLC, for the U. S. Department of Energy.

W.~Bangerth, T.~Heister, R.~Gassm{\"o}ller, and J.~Zhang were partially
supported by the Computational Infrastructure for Geodynamics initiative
(CIG), through the National Science Foundation (NSF) under Award
No.~EAR-1550901 and The University of California -- Davis.

W.~Bangerth and M.~Fehling were partially supported by Award OAC-1835673
as part of the Cyberinfrastructure for Sustained Scientific Innovation (CSSI)
program.

W.~Bangerth was also partially supported by Awards DMS-1821210 and EAR-1925595.

B.~Blais was partially supported by the National Science and Engineering
Research Council of Canada (NSERC)  through the RGPIN-2020-04510 Discovery
Grant

T.~Heister and J.~Zhang were also partially supported by NSF
Award OAC-2015848.

T.~Heister was also partially supported by the NSF Awards DMS-2028346,
EAR-1925575, and by Technical Data Analysis, Inc. through US Navy STTR
Contract N68335-18-C-0011.

R.~Gassm{\"o}ller was also partially supported by the NSF Award
EAR-1925595.

L.~Heltai was partially supported by the Italian Ministry of Instruction,
University and Research (MIUR), under the 2017 PRIN project NA-FROM-PDEs MIUR
PE1, ``Numerical Analysis for Full and Reduced Order Methods for the efficient
and accurate solution of complex systems governed by Partial Differential
Equations''.

M.~Kronbichler and P.~Munch were partially supported by the
Bayerisches Kompetenznetzwerk
f\"ur Technisch-Wissen\-schaft\-li\-ches Hoch- und H\"ochstleistungsrechnen
(KONWIHR) in the context of the projects
``Performance tuning of high-order discontinuous Galerkin solvers for
SuperMUC-NG'' and ``High-order matrix-free finite element implementations with
hybrid parallelization and improved data locality''.

M.~Maier was partially supported by NSF Awards DMS-1912847 and DMS-2045636.

D.~Wells was supported by NSF through Grant DMS-1344962.

The Interdisciplinary Center for Scientific Computing (IWR) at Heidelberg
University has provided hosting services for the \dealii{} web page.

\bibliography{paper}{}
\bibliographystyle{abbrv}

\end{document}
