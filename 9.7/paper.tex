\documentclass{ansarticle-preprint}
%\usepackage{ucs}
\usepackage[utf8]{inputenc}
\usepackage{amsmath}
%\usepackage{cite}
\usepackage{anslistings}
\usepackage{multicol}
\usepackage{pdfsync}
\usepackage{enumitem}

\usepackage{pgfplots}
\usepackage{pgfplotstable}

\usepackage{fontenc}
\usepackage{graphicx}
\usepackage{xspace}

\usepackage{siunitx}

\usepackage{floatflt}

\usepackage{multirow}

\usepackage{booktabs}

%\renewcommand{\baselinestretch}{2.0}
%\usepackage{lineno}
%\renewcommand\linenumberfont{\normalfont\tiny}
%\linenumbers

\graphicspath{{svg/}}

\usepackage[normalem]{ulem}

\usepackage{caption}
\usepackage{subcaption}

\usepackage{todonotes}

\pgfplotsset{compat=1.9}
\definecolor{gnuplot@lightblue}{RGB}{87,181,232}
\definecolor{gnuplot@green}{RGB}{0,158,115}
\definecolor{gnuplot@purple}{RGB}{148,0,212}

\newcommand{\specialword}[1]{\texttt{#1}}
\newcommand{\dealii}{{\specialword{deal.II}}\xspace}
\newcommand{\pfrst}{{\specialword{p4est}}\xspace}
\newcommand{\trilinos}{{\specialword{Trilinos}}\xspace}
\newcommand{\aspect}{\specialword{Aspect}\xspace}
\newcommand{\petsc}{\specialword{PETSc}\xspace}
\newcommand{\snes}{{\specialword{SNES}}\xspace}
\newcommand{\ts}{{\specialword{TS}}\xspace}
\newcommand{\petscsf}{{\specialword{SF}}\xspace}
\newcommand{\cmake}{{\specialword{CMake}}\xspace}
\newcommand{\candi}{{\specialword{candi}}\xspace}
\newcommand{\MPI}{{\specialword{MPI}}\xspace}
\newcommand{\MPIx}{{\specialword{MPI+X}}\xspace}
\newcommand{\sundials}{{\specialword{SUNDIALS}}\xspace}
\newcommand{\kinsol}{{\specialword{KINSOL}}\xspace}
\newcommand{\ida}{{\specialword{IDA}}\xspace}
\newcommand{\arkode}{{\specialword{ARKODE}}\xspace}
\newcommand{\boost}{{\specialword{Boost}}\xspace}
\newcommand{\kokkos}{{\specialword{Kokkos}}\xspace}
\newcommand{\llvm}{{\specialword{LLVM}}\xspace}
\newcommand{\step}[1]{{\specialword{step-#1}}\xspace}

%Trilinos Packages
\newcommand{\epetra}{{\specialword{Epetra}}\xspace}
\newcommand{\petra}{{\specialword{Petra}}\xspace}
\newcommand{\teuchos}{{\specialword{Teuchos}}\xspace}
\newcommand{\tpetra}{{\specialword{Tpetra}}\xspace}


\usetikzlibrary{shapes.misc}
\tikzset{cross/.style={cross out, draw=black, minimum size=2*(#1-\pgflinewidth), inner sep=0pt, outer sep=0pt},
%default radius will be 1pt.
cross/.default={2pt}}

%
% Author list -- please add yourself in both places below (in
%                alphabetical order) if you think that your
%                contributions to the last release warrant this
%

\hypersetup{
  pdfauthor={
    Daniel Arndt,
    Wolfgang Bangerth,
    Bruno Blais,
    Marc Fehling,
    Rene Gassm\"{o}ller,
    Timo Heister,
    Luca Heltai,
    Martin Kronbichler,
    Matthias Maier,
    Peter Munch,
    Bruno Turcksin,
    David Wells,
  },
  pdftitle={The deal.II Library, Version 9.7, 2025},
}

\title{The \dealii Library, Version 9.7}

 \author[1*]{Daniel Arndt}
 \affil[1]{Computational Coupled Physics Group,
   Computational Sciences and Engineering Division,
   Oak Ridge National Laboratory, 1 Bethel Valley Rd.,
   TN 37831, USA.
   \texttt{arndtd/turcksinbr@ornl.gov}}

 \author[2,3]{Wolfgang~Bangerth}
 \affil[2]{Department of Mathematics, Colorado State University, Fort
   Collins, CO 80523-1874, USA.
   \texttt{bangerth@colostate.edu}}
 \affil[3]{Department of Geosciences, Colorado State University, Fort
   Collins, CO 80523, USA.}

\author[4]{Bruno Blais}
\affil[4]{Chemical Engineering High-performance Analysis, Optimization and Simulation (CHAOS) laboratory, Department of Chemical Engineering,
             Polytechnique Montréal,
             PO Box 6079, Stn Centre-Ville, Montréal, Québec, Canada, H3C 3A7.
             {\texttt{bruno.blais@polymtl.ca}}}

\author[5]{Marc~Fehling}
 \affil[5]{Department of Mathematical Analysis,
    Faculty of Mathematics and Physics, Charles University,
    Sokolovsk{\'a} 49/83, 186\,75 Prague 8, Czech Republic.
    {\texttt{marc.fehling@matfyz.cuni.cz}}}


\author[6]{Rene~Gassm\"{o}ller}
\affil[6]{GEOMAR Helmholtz Centre for Ocean Research Kiel, 24148 Kiel, Germany}

\author[7]{Timo~Heister}
 \affil[7]{School of Mathematical and Statistical Sciences,
   Clemson University,
   Clemson, SC, 29634, USA.
   {\texttt{heister@clemson.edu}}}

\author[8]{Luca~Heltai}
\affil[8]{Department of Mathematics, University of Pisa,
Via Buonarroti 1/c, 56127 Pisa, Italy.}

 \author[9]{Martin~Kronbichler}
 \affil[9]{Faculty of Mathematics, Ruhr University Bochum,
   Universit\"atsstr.~150, 44780 Bochum, Germany.
 {\texttt{martin.kronbichler@rub.de}}}

\author[10]{Matthias~Maier}
\affil[10]{Department of Mathematics,
  Texas A\&M University,
  3368 TAMU,
  College Station, TX 77845, USA.
  {\texttt{maier@math.tamu.edu}}}

\author[11]{Peter Munch}
 \affil[11]{Institute of Mathematics, Technical University of Berlin, Germany.
  {\texttt{muench@math.tu-berlin.de}}}

\author[1*]{Bruno~Turcksin}

\author[12]{David Wells}
\affil[12]{Department of Mathematics, University of North Carolina,
  Chapel Hill, NC 27516, USA.
  {\texttt{drwells@email.unc.edu}}}

\renewcommand{\labelitemi}{--}


\begin{document}
\maketitle

\footnotetext{%
  $^\ast$ This manuscript has been authored by UT-Battelle, LLC under Contract No.
  DE-AC05-00OR22725 with the U.S. Department of Energy.
  % We need to submit the manuscript with the text below. If the editor
  % complains we can remove it.
  The United States
  Government retains and the publisher, by accepting the article for
  publication, acknowledges that the United States Government retains a
  non-exclusive, paid-up, irrevocable, worldwide license to publish or reproduce
  the published form of this manuscript, or allow others to do so, for United
  States Government purposes. The Department of Energy will provide public
  access to these results of federally sponsored research in accordance with the
  DOE Public Access Plan (http://energy.gov/downloads/doe-public-access-plan).
}


\begin{abstract}
  This paper provides an overview of the new features of the finite element
  library \dealii, version 9.7.
\end{abstract}



%%%%%%%%%%%%%%%%%%%%%%%%%%%%%%%%%%%%%%%%%%%%%%%%%%%%%%%%%%%%%%%%%%%%%%%%%%%%%%%%
%%%%%%%%%%%%%%%%%%%%%%%%%%%%%%%%%%%%%%%%%%%%%%%%%%%%%%%%%%%%%%%%%%%%%%%%%%%%%%%%
%%%%%%%%%%%%%%%%%%%%%%%%%%%%%%%%%%%%%%%%%%%%%%%%%%%%%%%%%%%%%%%%%%%%%%%%%%%%%%%%
\section{Overview}

\dealii version 9.7.0 was released July 22, 2025.
This paper provides an
overview of the new features of this release and serves as a citable
reference for the \dealii software library version 9.7. \dealii is an
object-oriented finite element library used around the world in the
development of finite element solvers. It is available for free under the
terms of the \emph{GNU Lesser General Public License} (LGPL). The \dealii
project is in the process of relicensing the library under the terms of
the \emph{Apache License 2.0 with LLVM Exception}. Downloads are
available at \url{https://www.dealii.org/} and
\url{https://github.com/dealii/dealii}.

The major changes of this release are:
%
\begin{itemize}
\item
\todo[inline]{Keep this list of major changes up-to-date - the formatting is the
  same as the 9.6 paper.}

\item \texttt{MappingP1}, a more efficient mapping for simplex elements
  (Section~\ref{subsec:mappingp1})

\item Improved support for transferring solution vectors during mesh refinement
  (Section~\ref{subsec:solutiontransfer})

\item Improved matrix-free support for multi-component systems as well as
  improved portability between GPUs via \texttt{Portable::MatrixFree}
  (Section~\ref{subsec:matrixfree})

\item Improved support for threading via TaskFlow, as well as support for
  several new external libraries and improved support for Kokkos and ArborX
  (Section~\ref{subsec:external})

\item Support for C++20 modules (Section~\ref{subsec:modules})

\item Three new tutorial programs and a new code gallery program
  (Section~\ref{subsec:steps})
\end{itemize}
%

While all of these major changes are discussed in detail in
Section~\ref{sec:major}, there
are a number of other noteworthy changes in the current \dealii release,
which we briefly outline in the remainder of this section:
%
\begin{itemize}
\item An implementation of the iterative linear solver, named
  \texttt{SolverMPGMRES}, has been added. The MPGMRES solver is variant of the
  widely used generalized minimal residual method (GMRES)
  \cite{Saad1986gmres}, distinguished by the support of $N$ different
  preconditioners during the course of the solution process
  \cite{Greif2016gmres}. Multiple Krylov spaces are constructed by
  $N$-variate, non-commuting polynomials of the preconditioners and the system
  matrix applied to a residual of fixed degree. The implementation in \dealii
  supports two strategies, a full variant that constructs all possible
  combinations of preconditioner application to the initial residual $r$,
  which for two preconditioners $P_1, P_2$ looks as
  \[
    r, P_1r, P_2r, P_1AP_1r, P_2AP_1r, P_1AP_2r, P_2AP_2r, P_1AP_1AP_1r,
   P_2AP_1AP_1r, \ldots, P_2AP_2AP_2r, \ldots,
 \]
 and a truncated version where no combinations of preconditioners are made, e.g.
 \[
    r, P_1r, P_2r, P_1AP_1r, P_2AP_2r, P_1AP_1AP_1r,
   P_2AP_2AP_2r, \ldots.
 \]
 The new solver is beneficial when two different preconditioners can help to
 more quickly span the space where the solution is expected.
\item A common operation in the finite element codes is to ask whether
  a specific shape function belongs to a particular solution
  variable. In the past, this was implemented by using the
  \texttt{FiniteElement::system\_to\_component\_index()} function that
  returns, among other information, which vector component a shape
  function belongs to. One then had to test whether this is the vector
  component (or one of the vector components) that corresponds to a
  specific variable. The new
  \texttt{FiniteElement::shape\_function\_belongs\_to()} makes this
  easier: It takes as argument a \texttt{FiniteElementExtractor} that
  corresponds to a scalar, vector-valued, or tensor-valued set of
  solution variables, and directly returns \texttt{true} or
  \texttt{false}.
\item Most of the functions in the \texttt{GridGenerator} namespace correspond
  to simple geometries (such as cubes, spheres, and other shapes with analytical
  formulas) discretized with hypercubes. The function
  \texttt{GridGenerator::convert\_hypercube\_to\_simplex\_mesh()} converts such
  discretizations to instead use simplices. That function now supports
  anisotropic splits, including splitting quadrilaterals into two triangles and
  hexahedra into six tetrahedra, in addition to the standard isotropic splits.
  These anisotropic splits significantly reduce the total number of cells and
  are useful in contexts where it is important to add as few new cells as
  possible.
\item The build system now supports interprocedural and link-time optimization
  (LTO). This can be enabled with the CMake configuration option
  \texttt{-DDEAL\_II\_WITH\_LTO=ON}. This option enables whole-program
  optimization at link-time and can make executables significantly faster.
\item \texttt{GridIn::read\_vtk()} now loads cell-centered data vectors stored
  in the input file and makes them available via
  \texttt{GridIn::get\_cell\_data()}.
\end{itemize}
%
The
\href{https://dealii.org/current/doxygen/deal.II/changes_between_9_6_0_and_9_7_0.html}{changelog}
-- listing more than 120
features and bugfixes --
contains a complete record of all changes; see \cite{changes97}.



%%%%%%%%%%%%%%%%%%%%%%%%%%%%%%%%%%%%%%%%%%%%%%%%%%%%%%%%%%%%%%%%%%%%%%%%%%%%%%%%
%%%%%%%%%%%%%%%%%%%%%%%%%%%%%%%%%%%%%%%%%%%%%%%%%%%%%%%%%%%%%%%%%%%%%%%%%%%%%%%%
%%%%%%%%%%%%%%%%%%%%%%%%%%%%%%%%%%%%%%%%%%%%%%%%%%%%%%%%%%%%%%%%%%%%%%%%%%%%%%%%
\section{Major changes to the library}
\label{sec:major}

This release of \dealii contains a number of large and significant changes,
which we will discuss in this section.


%%%%%%%%%%%%%%%%%%%%%%%%%%%%%%%%%%%%%%%%%%%%%%%%%%%%%%%%%%%%%%%%%%%%%%%%%%%%%%%%
\subsection{The \texttt{MappingP1} class}
\label{subsec:mappingp1}

Classes derived from the \texttt{Mapping} base class implement
the transformation from a reference cell (such as the unit
square or reference triangle) to a concrete cell that is part of a mesh.
\dealii has supported simplex and mixed meshes since the 9.3
release~\cite{dealii93}. That release included \texttt{MappingFE}, a
\texttt{Mapping} class that uses the shape functions of a finite
element to implement this transformation; many common mappings can be
written in this way, and the class works with a variety of different elements by
delegating most computations to a provided \texttt{FiniteElement}
object. For example, the common bi-/trilinear mappings for quadrilaterals and
hexahedra, as well as the linear mappings on triangles and simplices,
can be implemented using \texttt{MappingFE} along with the
\texttt{FE\_Q} or \texttt{FE\_SimplexP} classes.

The current
release includes a new \texttt{MappingP1} class, which contains an optimized
implementation of the standard affine simplex mapping (i.e., the mapping from
the unit/reference cell to the real cell with a constant Jacobian on each cell). This
implementation is about six times faster than the generic one implemented in
\texttt{MappingFE}. In addition, this \texttt{Mapping} is now the default linear
mapping returned by \texttt{ReferenceCell::get\_default\_linear\_mapping()}, so
all applications which use that function for obtaining a
\texttt{ReferenceCell}-independent mapping will automatically use the more
performant class.

\subsection{The \texttt{SolutionTransfer} class}
\label{subsec:solutiontransfer}
This release extends the \texttt{SolutionTransfer} class to facilitate
the transfer of solution vectors from one mesh to another in an
adaptive mesh refinement loop, now to
all triangulation classes that support mesh refinement.
Consequently, it has subsumed the functionality of the existing
\texttt{parallel::distributed::SolutionTransfer} class, which is no
longer necessary and is now deprecated.

In the process of this update, we have also cleaned up these classes'
interfaces. In particular, some member functions, such as
\texttt{execute\_pure\_refinement()}, have been removed from the new
implementation because they provided functionality for only rarely
used cases, but substantially complicated the internal design of these
classes.


\subsection{Matrix-free functionality}
\label{subsec:matrixfree}

In this release, several extensions, major bug fixes and optimizations of the
matrix-free infrastructure and related multigrid functionality, such as
matrix-free geometric and polynomial coarsening, in \dealii have been
made. The most significant advances were made for the \kokkos-based
``portable'' infrastructure described below. In addition, improvements where
made to several areas:
\begin{itemize}
\item the \texttt{MatrixFreeTools::compute\_matrix()} and
  \texttt{MatrixFreeTools::compute\_diagonal()} functions to compute the
  matrix or only its diagonal from a matrix-free operator were enhanced
  towards multi-component systems, the DG context (and, in general, to operators
  with face integrals), and mixed elements,
\item support for additional ghost cells in the matrix-free setup was added to
  support vector access to a wider range operations, such as MPI
  parallelization of multigrid smoothers based on vertex
  patches~\cite{Wichrowski2025smoothers},
\item functionality for identifying degrees of freedom in lexicographic
  ordering on patches of cells,
\item support for additional matrices in the
  \texttt{TensorProductMatrixCreator} for support in fast
  diagonalization-based inverses.
\end{itemize}

\subsubsection{Portable::Matrix-free}
\texttt{Portable::MatrixFree} utilizes \kokkos to provide support for GPUs from
Nvidia, AMD, and Intel. In this release, we have added support for finite
elements with multiple components, i.e vector-values finite elements. We have
also added support for finite elements where the number
of quadrature points per direction is different from the size of the
one-dimensional basis (i.e., polynomial degree plus one), thus supporting more
cases of evaluation. In addition to the extended functionality, several
performance optimizations to the evaluation routines have been made, which are
particularly visible on recent Nvidia GPUs.

To ensure more uniform handling of temporary data structures necessary for
storing temporary results in matrix-free evaluations, such as
sum-factorization algorithms, we have introduced two new classes:
\texttt{Portable::DeviceVector} and
\texttt{Portable::DeviceBlockVector}. The class \texttt{Portable::DeviceVector} is an
alias for \texttt{Kokkos::View} that replaces the raw pointer previously used
to access data on the GPU. The class \texttt{Portable::DeviceBlockVector} extends
\texttt{Portable::DeviceVector} in a manner similar to the way
\texttt{BlockVector} extends \texttt{Vector} on the CPU.

Finally, we have streamlined the interface of the functor invoked by the GPU
kernel. In the past, this interface included a cell index,
\texttt{Portable::MatrixFree::Data}, which includes precomputed data such as
quadrature points, shape functions, and other informations that needed to be
transferred to the GPU, and \texttt{Portable::MatrixFree::SharedData} which
contained scratch memory used for computation. Now, all this information has
been consolidated into a single new structure:
\texttt{Portable::MatrixFree::Data}, greatly simplifying the interface.

%% The major simplex additions are the changes to
%% convert_hypercube_to_simplex_mesh() and MappingP1: those are discussed
%% elsewhere so we don't need a dedicated section for the general topic

\subsection{Updates of \dealii{}'s interfaces to external libraries}
\label{subsec:external}

\dealii{} relies on external software packages for many operations
that are not within the core functionality of a finite element package
-- see Section~\ref{sec:cite} for a complete list. As in every
release, we have continued to expand and revise these interfaces. The
following sub-sections outline these updates.

\subsubsection{Interfaces to the Threading Building Blocks and to
  Taskflow}

A large number of operations within \dealii{} are parallelized,
including through mechanisms like WorkStream \cite{TKB16} that are
also available to and useful in application codes. For more than
15 years, since the 6.3 release, \dealii{} has used the Threading
Building Blocks (TBB) \cite{reinders2007tbb} as the underlying library to
provide task- and pipeline-based parallelism. However, the TBB has
downsides that primarily follow from its use of assembler mnemonics
and the resulting difficulties with supporting all platforms and
compilers. As a consequence, starting with the current release,
\dealii{} builds on the Taskflow library \cite{huang2021taskflow} that
is entirely implemented in C++, is header-only, and provides an
interface that matches well with current language standards. \dealii{}
bundles the Taskflow 3.10 release, and now no longer bundles the TBB,
though the TBB interfaces are still available if a TBB installation
is found on a system.

\todo[inline]{Timo: Perhaps add the figures from the WorkStream paper
  updated for TBB vs Taskflow.}


\subsubsection{\kokkos, a performance portability library}
\label{subsec:external-kokkos}

\kokkos is a C++ library that makes it easy to write codes that can
portably run on CPUs and GPUs of different vendors, using their
respective underlying programming models.
The \dealii{} 9.5 release~\cite{dealII95} made \kokkos~\cite{trott2022} a required
dependency. The current release removes the deprecated \texttt{CUDAWrappers} namespace as well
as all \texttt{CUDA}-related macros in favor of \kokkos
equivalents. The 9.7 release now bundles \kokkos 4.5.1 but the minimum version required is still 3.4.

\subsubsection{MUMPS, a multifrontal parallel direct solver}

MUMPS (MUltifrontal Massively Parallel sparse direct Solver) is a widely-used software library designed for solving large sparse systems of linear equations efficiently on parallel architectures~\cite{amestoy2001mumps}.

In the 9.7 release, we have significantly enhanced integration with MUMPS,
without requiring it to be installed through PETSc or Trilinos. Improvements
include a simplified user interface for direct solvers, mimicking the existing
interface we provide for UMFPACK, and enhanced performance optimizations that
leverage recent MUMPS capabilities. Specifically, the deal.II wrappers now
transparently support parallel matrix factorization, out-of-core computations,
Block Low-Rank approximations, and GPU support.

\subsubsection{PSBLAS, a parallel sparse linear algebra library}

PSBLAS is a parallel sparse linear algebra library that provides basic operations for sparse matrices and vectors~\cite{Filippone2000psblas}.

The library is specifically designed to leverage modern
parallel computing architectures, including both CPU and GPU systems, making it well-suited for exascale
computing environments. Release 9.7 includes preliminary support for configuring with PSBLAS, allowing users of both frameworks to take advantage of its capabilities for specific linear algebra tasks. This support is still in its early stages, and the library is not yet fully integrated into the \dealii{} ecosystem.

\subsubsection{\texttt{magic\_enum}, a static reflection library for enums}

\texttt{magic\_enum} is a modern, header-only C++17 library designed to provide static reflection capabilities for enums without the need for macros or additional boilerplate code. It greatly simplifies working with enum types by enabling conversions between enum values and strings, integer values, and various other operations. Version 9.7 uses \texttt{magic\_enum} to provide type safe and efficient conversions between enum values and their string representations in the \texttt{ParameterHandler} class.


\todo[inline]{Peter @ Luca: I think a code snippet would be great here! }

\subsubsection{ArborX, a performance-portable geometric search library}
ArborX~\cite{prokopenko2025} is a library for geometric search on CPUs and GPUs.
The new ArborX 2.X series is not backward compatible with the older 1.X series.
Consequently, this incompatibility required an extensive rewrite of our wrappers to
support ArborX 2.X. As part of the rewrite, the template parameters had to
be changed. When using the wrappers with ArborX 1.X, \texttt{ArborX::BVH}
and \texttt{ArborX::DistributedTree} are templated on both the dimension and
the number types. When using the wrappers with ArborX 2.X, these classes are
templated on the geometric objects used for the leaf nodes of the
\texttt{ArborX::BVH}.


\subsection{Work towards C++20 modules}
\label{subsec:modules}

\dealii{} currently requires a compiler that understands C++17, but
some additional features are enabled if a compiler supports the C++20
standard. C++20 also introduces ``modules'', a way by which software
packages can export their public interface without relying on the
mechanism of textual inclusion of header files that C++ has inherited from C. The
end goal of C++20 modules is that one can replace the use of long
lists of statements such as\\
\hspace*{1cm}\texttt{\#include <deal.II/grid/tria.h>}\\
by a simple\\
\hspace*{1cm}\texttt{import dealii;}

A lot of effort -- perhaps 6 weeks of full-time work, and nearly 200
pull requests -- has gone into evaluating whether \dealii{} can be
built using modules (in addition to header files, which will need to
be provided for a long time to come for backward compatibility). A
detailed accounting of this effort can be found in
\cite{bangerth2025experienceconvertinglargemathematical}. The majority
of this work addressed features of the \dealii{} code base,
accumulated over its more than 25 year history, that were acceptable
in a header-based system but not in a module-based system. To name
just two of many examples: (i) there can be cycles of header files that mutually
\texttt{\#include} each other, but this is not allowed for
modules; and (ii) it is allowed to declare functions in header files
as \texttt{static} or in anonymous namespaces, but this no longer
works with modules. In both cases, while allowed, the existing code
was brittle, inefficient, or hard to understand. Most of the pull requests for this project therefore simply
cleaned up our code base, without having to introduce any
incompatibilities.

As shown in \cite{bangerth2025experienceconvertinglargemathematical},
the use of C++20 modules
reduces compile times for the library itself, though the evidence is
mixed for tutorial programs or applications that build on
\dealii{}. Given that at the moment few systems have compilers, CMake
versions, and build systems that are sufficiently new to support
module builds, it will be several years before this feature will
become widely available. However, the infrastructure for it is now
largely in place.



%%%%%%%%%%%%%%%%%%%%%%%%%%%%%%%%%%%%%%%%%%%%%%%%%%%%%%%%%%%%%%%%%%%%%%%%%%%%%%%%
\subsection{New and improved tutorials and code gallery programs}
\label{subsec:steps}

Many of the \dealii tutorial programs were revised in a variety of ways
as part of this release: Around 125 of the more than 1500 (non-merge)
commits that went into this release touched the tutorial.
% data generated using these commands:
% - tutorial: git log --since 2024/08/11 --until 2025/07/19 --no-merges examples | grep commit | wc -l
% - total:    git log --since 2024/08/11 --until 2025/07/19 --no-merges | grep commit | wc -l
In addition, there are a number of new tutorial
programs:
\begin{itemize}
  \item
    \step{93}
    demonstrates how one can implement problems in which some of the
    variables to be solved for are not degrees of freedom of a finite
    element field (so-called ``non-local'' degrees of freedom, because
    they have no associated node functional that is tied to a cell or
    a patch of cells). Specifically, the program solves an
    optimization problem in which some of the variables are scalars
    that multiply certain source terms.
    \step{93} was written by Sam Scheuerman.
  \item
    \step{95} extends \step{85} and illustrates the use of matrix-free methods to solve
    problems with embedded boundaries using the CutFEM method~\cite{bergbauer2025high}. \step{95} was written
    by Maximilian Bergbauer, with help by Martin Kronbichler and Peter Munch.
  \item
    \step{97}
    is a program that solves a curl-curl problem from electromagnetics
    -- more specifically, from magnetostatics. The program shows a
    formulation in which two curl-curl problems are solved
    consecutively to model the magnetic field that results from the
    current in an electrically conducting coil.
    \step{97} was written by Siarhei Uzunbajakau.
\end{itemize}

In addition, there is one new program in the code gallery (a collection of
user-contributed programs that often solve more complicated problems
than tutorial programs, and that are intended as starting points for further
research rather than as teaching tools):
\begin{itemize}
  \item \emph{``Phase field fracture model in 3D''},
    contributed by Wasim Niyaz Munshi,
    Chandrasekhar Annavarapu,
    Wolfgang Bangerth, and
    Marc Fehling.

\todo[inline]{Peter: A short description would be great here!}
\end{itemize}


%%%%%%%%%%%%%%%%%%%%%%%%%%%%%%%%%%%%%%%%%%%%%%%%%%%%%%%%%%%%%%%%%%%%%%%%%%%%%%%%
\subsection{Incompatible changes}\label{subsec:deprecated}

The 9.7 release includes
\href{https://dealii.org/developer/doxygen/deal.II/changes_between_9_6_0_and_9_7_0.html}
     {around 25 incompatible changes};
see \cite{changes97}. Many of these
incompatibilities remove previously deprecated functions or classes,
or require now widely available but newer versions of external
dependencies than previous \dealii{} versions. Others
change internal
interfaces that are not usually used in external
applications. The following are worth mentioning since they
are more broadly visible:
\begin{itemize}
  \item The classes \texttt{Subscriptor} and \texttt{SmartPointer} have been
    renamed to \texttt{EnableObserverPointer} and \texttt{ObserverPointer}. Many
    core classes, such as \texttt{DoFHandler} and \texttt{Triangulation},
    inherit from \texttt{EnableObserverPointer} to enable other classes to store
    \texttt{ObserverPointer}s pointing to them which, if dereferenced after the
    pointed-to object goes out of scope, will signal an error. This
    functionality has been in \dealii since 1998. Since ``smart pointers''
    nowadays means \texttt{std::unique\_ptr} or \texttt{std::shared\_ptr}, these
    names were updated to avoid this conflict in terminology. The new names are
    based on the proposal for \texttt{std::observer\_ptr}~\cite{Brown2014} and
    \texttt{std::enable\_shared\_from\_this}.

  \item In order to support modules (see Section~\ref{subsec:modules}), a
    significant number of headers had to be split up to prevent circular
    inclusions and other minor incompatibilities. For example,
    \texttt{grid/tria.h} no longer contains declarations of \texttt{CellData} or
    \texttt{SubCellData}. Those declarations have been moved to
    \texttt{grid/cell\_data.h}. In cases where this leads to errors,
    the situation is easily (and backward compatibly) fixed by adding an explicit
    \texttt{\#include} statement to the user program for the file that is necessary to
    access declarations the compiler now no longer sees.

  \item As already mentioned in
    Section~\ref{subsec:external-kokkos}, the current release
    removes the deprecated \texttt{CUDAWrappers} namespace as well
    as all \texttt{CUDA}-related macros in favor of \kokkos
    equivalents.

    % This is similar to a note in the 9.6 paper, but I (David) made more
    % orientation changes for 9.7. These were done in #17988 and #18333.
  \item We continued unifying the various implementations of object
    orientations in the library. While the three booleans orientation, rotation,
    and flip have maintained their original definitions (i.e., they have default
    values of \texttt{true}, \texttt{false}, and \texttt{false}), the default
    value of the combined orientation (available via
    \texttt{TriaAccessor::combined\_face\_orientation()} and
    \texttt{TriaAccessor::line\_orientation()}) is now $0$ rather than $1$. In
    addition, the return type of these functions is now
    \texttt{types::geometric\_orientation}, which is a \texttt{typedef} for an
    8-bit integral type.

  \item The \step{52} tutorial was removed. This program showed
    the use of time stepping functionality in \dealii{}, but
    this functionality was rudimentary compared to what packages such
    as SUNDIALS or PETSc TS offer. As a consequence, \step{52} did
    not show the advanced techniques we would like to promote (e.g.,
    time stepping error control and automatic time step size choice), and it
    is now superseded by \step{86} (which builds on PETSc TS).
  \item The \texttt{parallel::distributed::SolutionTransfer} class has been
    merged with the class \texttt{SolutionTransfer}, with the
    former now being deprecated. In the process of cleaning up the
    interface, we have also removed some rarely used member functions
    of the latter that prevented us from unifying the two classes. For more
    information see Section~\ref{subsec:solutiontransfer}.
  \item The \texttt{MatrixOut} class, which creates graphical
    representations of matrices, now defaults to creating sparse
    output, only showing nonzero entries. This vastly increases the
    size of matrices that can be visualized.
  \item The \texttt{FiniteElement} class has a number of functions
    that allow querying properties of implementations in derived
    classes. These include asking whether a derived class is able to provide
    ``(generalized) support points'', i.e., whether the element
    defines its shape functions via nodal interpolation (or via
    quadrature-based integrals) and can return an array of these
    points. These functions returned \texttt{false} for the
    \texttt{FE\_Nothing} element -- an element that has no degrees of
    freedom and consequently no shape functions. This answer was not
    wrong, but missed the point: Codes using these functions want to
    know whether an element is able to provide arrays with these
    support points, and for \texttt{FE\_Nothing} the answer should be
    ``yes'' (i.e., \texttt{true}): The class \textit{can} provide such
    an array, it will just be empty.
\end{itemize}



%%%%%%%%%%%%%%%%%%%%%%%%%%%%%%%%%%%%%%%%%%%%%%%%%%%%%%%%%%%%%%%%%%%%%%%%%%%%%%%%
%%%%%%%%%%%%%%%%%%%%%%%%%%%%%%%%%%%%%%%%%%%%%%%%%%%%%%%%%%%%%%%%%%%%%%%%%%%%%%%%
%%%%%%%%%%%%%%%%%%%%%%%%%%%%%%%%%%%%%%%%%%%%%%%%%%%%%%%%%%%%%%%%%%%%%%%%%%%%%%%%
\section{How to cite \dealii}
\label{sec:cite}

In order to justify the work the developers of \dealii put into this
software, we ask that papers using the library reference one of the
\dealii papers. This helps us justify the effort we put into this library.

There are various ways to reference \dealii. To acknowledge the use of
the current version of the library, \textbf{please reference the present
  document}. For up-to-date information and a bibtex entry
see
\begin{center}
  \url{https://www.dealii.org/publications.html}
\end{center}

The original \dealii paper containing an overview of its
architecture is \cite{BangerthHartmannKanschat2007}, and a more recent
publication documenting \dealii's design decisions is available as \cite{dealII2020design}. If you rely on
specific features of the library, please consider citing any of the
following:
\begin{multicols}{2}
  \vspace*{-36pt}
  \begin{itemize}[leftmargin=4mm]
    \item For geometric multigrid: \cite{Kanschat2004,JanssenKanschat2011,ClevengerHeisterKanschatKronbichler2019, munch2022gc};
    \item For distributed parallel computing: \cite{BangerthBursteddeHeisterKronbichler11};
    \item For $hp$-adaptivity: \cite{BangerthKayserHerold2007,fehlingbangerth2023};
    \item For partition-of-unity (PUM) and finite element enrichment methods:
           \cite{Davydov2016};
    \item For matrix-free and fast assembly techniques:
          \cite{KronbichlerKormann2012,KronbichlerKormann2019};
    \item For computations on lower-dimensional manifolds:
          \cite{DeSimoneHeltaiManigrasso2009};
    \item For curved geometry representations and manifolds:
          \cite{HeltaiBangerthKronbichlerMola2019};
    \item For integration with CAD files and tools:
          \cite{HeltaiMola2015};
    \item For boundary element computations:
          \cite{GiulianiMolaHeltai-2018-a};
    \item For the \texttt{LinearOperator} and
      \texttt{Packaged\-Operation} facilities:
          \cite{MaierBardelloniHeltai-2016-a,MaierBardelloniHeltai-2016-b};
    \item For uses of the \texttt{WorkStream} interface:
          \cite{TKB16};
    \item For uses of the \texttt{ParameterAcceptor} concept, the
          \texttt{MeshWorker::ScratchData} base class, and the
          \texttt{ParsedConvergenceTable} class:
          \cite{SartoriGiulianiBardelloni-2018-a};
    \item For uses of the particle functionality in \dealii:
      \cite{GLHPB18}.
      \item For the design of the video lectures and how they can be
        used in \dealii{}-based courses: \cite{Zarestky2022}.
          \vfill\null
  \end{itemize}
\end{multicols}

\dealii can interface with many other libraries:
\begin{multicols}{3}
  \begin{itemize}[leftmargin=4mm]
    \item ADOL-C \cite{griewank1996adolc}
    \item ArborX \cite{prokopenko2025}
    \item ARPACK \cite{lehoucq1998arpack}
    \item Assimp \cite{schulze2021assimp}
    \item BLAS and LAPACK \cite{anderson1999lapack}
    \item Boost \cite{boost-web-page}
    \item CGAL \cite{cgal:eb-24b}
    \item Gmsh \cite{geuzaine2009gmsh}
    \item GSL \cite{galassi2009gsl,gsl-web-page}
    \item Ginkgo \cite{anzt2020ginkgo,anzt2022ginkgo}
    \item HDF5 \cite{hdf5-web-page}
    \item Kokkos \cite{trott2022}
    \item Magic Enum \cite{magic-enum-web-page}
    \item METIS \cite{karypis1998metis}
    \item MUMPS \cite{amestoy2001mumps,amestoy2019mumps}
    \item muparser \cite{muparser-web-page}
    \item OpenCASCADE \cite{opencascade-web-page}
    \item p4est \cite{burstedde2011p4est}
    \item PETSc \cite{petsc-user-ref,petsc-web-page}
    \item PSBLAS \cite{Filippone2000psblas}
    \item ROL \cite{ridzal2014rol}
    \item ScaLAPACK \cite{blackford1997scalapack}
    \item SLEPc \cite{roman2023improvements}
    \item SUNDIALS \cite{gardner2022enabling}
    \item SymEngine \cite{symengine-web-page}
    \item Taskflow \cite{huang2021taskflow}
    \item TBB \cite{reinders2007tbb}
    \item Trilinos \cite{mayr2025trilinos,trilinos-web-page}
    \item UMFPACK \cite{davis2004umfpack}
    \item VTK \cite{vtk-web-page}
    \item zlib \cite{zlib-web-page}
  \end{itemize}
\end{multicols}
Please consider citing the appropriate references if you use
interfaces to these libraries.

The two previous releases of \dealii can be cited as
\cite{dealII95,dealII96}.


\section{Acknowledgments}

\dealii is a worldwide project with dozens of contributors around the
globe. Other than the authors of this paper, the following people
contributed code to this release:\\

\begin{quote}
%  authors removed:
%  Daniel       Arndt,
%  Wolfgang     Bangerth,
%  Bruno        Blais,
%  Marc         Fehling,
%  Rene         Gassmoeller,
%  Timo         Heister,
%  Luca         Heltai,
%  Martin       Kronbichler,
%  Matthias     Maier,
%  Peter        Muench,
%  Jean-Paul    Pelteret,
%  Bruno        Turcksin,
%  David        Wells,
Pasquale     Africa,
Henry        Arhin,
Maximilian   Bergbauer,
Alfredo      Buttari,
Bruna        Campos,
Nicholas     Cantrell,
Xiaoming     Cao,
Chayapol     Chaoveeraprasit,
Jerett       Cherry,
Dario        Coscia,
John         Coughlin,
Nikita       Daniliuk,
Crystal      Farris,
Marco        Feder,
Emmanuel     Ferdman,
Federico     Fernandez,
Olivier      Gaboriault,
Mohamad      Ghadban,
Mahdi        Gharehbaygloo,
Robin        Görmer,
Davit        Gyulamiryan,
Lóránt       Hadnagy,
Fernando     Herrera,
Robin        Hiniborch,
Quang        Hoang,
Jordan       Hoffart,
Sascha       Hofstetter,
Yimin        Jin,
Yann         Jobic,
Sean         Johnson,
Vaishnavi    Kale,
Sebastian    Kinnewig,
Andreas      Koch,
Jason        Landini,
Zhou         Lei,
Yingli       Li,
Nils         Margenberg,
Oreste       Marquis,
Ryan         Moulday,
Nils         Much,
Tileuzhan    Mukhamet,
Peter        Munch,
Wasim~Niyaz  Munshi,
Natalia      Nebulishvili,
Luz          Paz,
David        Pecoraro,
Davide       Polverino,
Sanjeeb      Poudel,
Guilhem      Poy,
Laura        Prieto~Saavedra,
Sebastian    Proell,
Andreas      Ritthaler,
Mayank       Sabharwal,
Sam          Scheuerman,
Magdalena    Schreter,
Richard      Schussnig,
Kyle         Schwiebert,
Marc         Secanell,
Qingyuan     Shi,
Wyatt        Smith,
Simon        Sticko,
Dominik      Still,
Edward       Terrell,
Jan~Philipp  Thiele,
Xiaochuan    Tian,
Siarhei      Uzunbajakau,
Mikael       Vaillant,
Vinayak      Vijay,
Stephan      Voss,
Michał       Wichrowski,
Simon        Wiesheier,
Yi-Yung      Yang,
\end{quote}
Their contributions are much appreciated!

This release contains patches written in and contributed from
Antarctica. As a consequence, \dealii{} now consists of work written
on all seven continents.

\bigskip

\dealii and its developers are financially supported through a
variety of funding sources:

\todo[inline]{Everyone please update as appropriate.}
D.~Arndt and B.~Turcksin: Research sponsored by the Laboratory Directed Research and
Development Program of Oak Ridge National Laboratory, managed by UT-Battelle,
LLC, for the U. S. Department of Energy and supported by the U.S. Department of Energy,
Office of Science, Office of Advanced Scientific Computing Research,
Next-Generation Scientific Software Technologies program, under contract
number DE-AC05-00OR22725.

W.~Bangerth was partially supported by the National Science Foundation
under awards OAC-1835673, EAR-1925595, and OAC-2410847.

W.~Bangerth, T.~Heister, and R.~Gassm\"{o}ller were partially
supported by the Computational Infrastructure for Geodynamics initiative
(CIG), through the National Science Foundation (NSF) under Award
No.~EAR-2149126 via The University of California -- Davis.


B.~Blais was supported by the National Science and Engineering Research Council of Canada (NSERC)  through the RGPIN-2020-04510 Discovery Grant and the MMIAOW Canada Research Level 2 in Computer-Assisted Design and Scale-up of Alternative Energy Vectors for Sustainable Chemical Processes.

M.~Fehling was partially supported by the ERC-CZ grant LL2105 CONTACT,
funded by the Czech Ministry of Education, Youth and Sports. He was also
partially supported by the Charles University Research Centre Program No.\@ UNCE/24/SCI/005.

R.~Gassm\"{o}ller was also partially supported by NSF Awards EAR-1925677
and EAR-2054605.

T.~Heister were also partially supported by NSF Awards OAC-2015848 and EAR-1925575.

T.~Heister was also partially supported by NSF OAC-2410848.

L.~Heltai is a member of Gruppo Nazionale per il Calcolo Scientifico (GNCS) of
Istituto Nazionale di Alta Matematica (INdAM). LH was partially supported by
the Italian Ministry of University and Research (MUR), under the grant MUR PRIN
2022 No. 2022WKWZA8 ``Immersed methods for multiscale and multiphysics problems
(IMMEDIATE)'', and acknowledges the MIUR Excellence Department Project awarded
to the Department of Mathematics, University of Pisa, CUP I57G22000700001.

M.~Kronbichler was partially supported by the
German Federal Ministry of Research, Technology and Space, project
``PDExa: Optimized software methods for solving partial differential
equations on exascale supercomputers'', grant agreement No. 16ME0637K.

M.~Kronbichler and L.~Heltai were partially supported by the European Research
Council (ERC) under the European Union's Horizon 2020 research and innovation
programme (call HORIZON-EUROHPC-JU-2023-COE-03, grant agreement No. 101172493
``dealii-X: an Exascale Framework for Digital Twins of the Human Body'').

M.~Maier was partially supported by NSF Award DMS-2045636 and by the
Air Force Office of Scientific Research under grant/contract number
FA9550-23-1-0007.

D.~Wells was supported by NSF Award OAC-1931516.

Clemson University is acknowledged for generous allotment of compute
time on the Palmetto cluster.
\todo{Still relevant?}

The authors acknowledge the Texas Advanced Computing Center (TACC) at The University of Texas at Austin for providing HPC resources that have contributed to the research results reported within this paper. \url{http://www.tacc.utexas.edu}
\todo{Still relevant?}

This work used the \textit{Expanse} HPC system at the San Diego Supercomputer Center (SDSC) at UC San Diego through the CIG Science Gateway and Community Codes for the Geodynamics Community MCA08X011 allocation from the Advanced Cyberinfrastructure Coordination Ecosystem: Services \& Support (ACCESS) program, which is supported by National Science Foundation grants \#2138259, \#2138286, \#2138307, \#2137603, and \#2138296. See \cite{Boerner2023}.
\todo{Still relevant?}


\bibliography{paper}{}
\bibliographystyle{abbrv}

\end{document}
