\documentclass{ansarticle-preprint}
%\usepackage{ucs}
\usepackage[utf8]{inputenc}
\usepackage{amsmath}
%\usepackage{cite}
\usepackage{anslistings}
\usepackage{multicol}
\usepackage{pdfsync}
\usepackage{enumitem}

\usepackage{pgfplots}
\usepackage{pgfplotstable}

\usepackage{fontenc}
\usepackage{graphicx}
\usepackage{xspace}

\usepackage{siunitx}

\usepackage{floatflt}

\usepackage{multirow}

\usepackage{booktabs}

%\renewcommand{\baselinestretch}{2.0}
%\usepackage{lineno}
%\renewcommand\linenumberfont{\normalfont\tiny}
%\linenumbers

\graphicspath{{svg/}}

\usepackage[normalem]{ulem}

\usepackage{caption}
\usepackage{subcaption}

\usepackage{todonotes}

\pgfplotsset{compat=1.9}
\definecolor{gnuplot@lightblue}{RGB}{87,181,232}
\definecolor{gnuplot@green}{RGB}{0,158,115}
\definecolor{gnuplot@purple}{RGB}{148,0,212}

\newcommand{\specialword}[1]{\texttt{#1}}
\newcommand{\dealii}{{\specialword{deal.II}}\xspace}
\newcommand{\pfrst}{{\specialword{p4est}}\xspace}
\newcommand{\trilinos}{{\specialword{Trilinos}}\xspace}
\newcommand{\aspect}{\specialword{Aspect}\xspace}
\newcommand{\petsc}{\specialword{PETSc}\xspace}
\newcommand{\snes}{{\specialword{SNES}}\xspace}
\newcommand{\ts}{{\specialword{TS}}\xspace}
\newcommand{\petscsf}{{\specialword{SF}}\xspace}
\newcommand{\cmake}{{\specialword{CMake}}\xspace}
\newcommand{\candi}{{\specialword{candi}}\xspace}
\newcommand{\MPI}{{\specialword{MPI}}\xspace}
\newcommand{\MPIx}{{\specialword{MPI+X}}\xspace}
\newcommand{\sundials}{{\specialword{SUNDIALS}}\xspace}
\newcommand{\kinsol}{{\specialword{KINSOL}}\xspace}
\newcommand{\ida}{{\specialword{IDA}}\xspace}
\newcommand{\arkode}{{\specialword{ARKODE}}\xspace}
\newcommand{\boost}{{\specialword{Boost}}\xspace}
\newcommand{\kokkos}{{\specialword{Kokkos}}\xspace}
\newcommand{\llvm}{{\specialword{LLVM}}\xspace}
\newcommand{\step}[1]{{\specialword{step-#1}}\xspace}

%Trilinos Packages
\newcommand{\epetra}{{\specialword{Epetra}}\xspace}
\newcommand{\petra}{{\specialword{Petra}}\xspace}
\newcommand{\teuchos}{{\specialword{Teuchos}}\xspace}
\newcommand{\tpetra}{{\specialword{Tpetra}}\xspace}


\usetikzlibrary{shapes.misc}
\tikzset{cross/.style={cross out, draw=black, minimum size=2*(#1-\pgflinewidth), inner sep=0pt, outer sep=0pt},
%default radius will be 1pt.
cross/.default={2pt}}

%
% Author list -- please add yourself in both places below (in
%                alphabetical order) if you think that your
%                contributions to the last release warrant this
%

\hypersetup{
  pdfauthor={
    Pasquale Claudio Africa,
    Daniel Arndt,
    Wolfgang Bangerth,
    Bruno Blais,
    Marc Fehling,
    Rene Gassm\"{o}ller,
    Timo Heister,
    Luca Heltai,
    Martin Kronbichler,
    Matthias Maier,
    Peter Munch,
    Magdalena Schreter-Fleischhacker,
    Jan Philipp Thiele,
    Bruno Turcksin,
    David Wells,
    Vladimir Yushutin
  },
  pdftitle={The deal.II Library, Version 9.7, 2025},
}

\title{The \dealii Library, Version 9.7}

 \author[1*]{Daniel Arndt}
 \affil[1]{Computational Coupled Physics Group,
   Computational Sciences and Engineering Division,
   Oak Ridge National Laboratory, 1 Bethel Valley Rd.,
   TN 37831, USA.
   \texttt{arndtd/turcksinbr@ornl.gov}}

 \author[2,3]{Wolfgang~Bangerth}
 \affil[2]{Department of Mathematics, Colorado State University, Fort
   Collins, CO 80523-1874, USA.
   \texttt{bangerth@colostate.edu}}
 \affil[3]{Department of Geosciences, Colorado State University, Fort
   Collins, CO 80523, USA.}

\author[4]{Bruno Blais}
\affil[4]{Chemical Engineering High-performance Analysis, Optimization and Simulation (CHAOS) laboratory, Department of Chemical Engineering,
             Polytechnique Montréal,
             PO Box 6079, Stn Centre-Ville, Montréal, Québec, Canada, H3C 3A7.
             {\texttt{bruno.blais@polymtl.ca}}}

\author[5]{Marc~Fehling}
 \affil[5]{Department of Mathematical Analysis,
    Faculty of Mathematics and Physics, Charles University,
    Sokolovsk{\'a} 49/83, 186\,75 Prague 8, Czech Republic.
    {\texttt{marc.fehling@matfyz.cuni.cz}}}


\author[6]{Rene~Gassm\"{o}ller}
\affil[6]{GEOMAR Helmholtz Centre for Ocean Research Kiel, 24148 Kiel, Germany}

\author[7]{Timo~Heister}
 \affil[7]{School of Mathematical and Statistical Sciences,
   Clemson University,
   Clemson, SC, 29634, USA.
   {\texttt{heister@clemson.edu}}}

\author[8]{Luca~Heltai}
\affil[8]{Department of Mathematics, University of Pisa, 
Via Buonarroti 1/c, 56127 Pisa, Italy.}

 \author[9,10]{Martin~Kronbichler}
 \affil[9]{Faculty of Mathematics, Ruhr University Bochum,
   Universit\"atsstr.~150, 44780 Bochum, Germany.
 {\texttt{martin.kronbichler@rub.de}}}
 \affil[10]{Institute of Mathematics,
   University of Augsburg,
   Universit\"atsstr.~12a, 86159 Augsburg, Germany.
   }

\author[11]{Matthias~Maier}
\affil[11]{Department of Mathematics,
  Texas A\&M University,
  3368 TAMU,
  College Station, TX 77845, USA.
  {\texttt{maier@math.tamu.edu}}}

\author[10,12]{Peter Munch}
 \affil[12]{Uppsala University, Sweden.
  {\texttt{peter.munch@it.uu.se}}}

\author[1*]{Bruno~Turcksin}

\author[13]{David Wells}
\affil[13]{Department of Mathematics, University of North Carolina,
  Chapel Hill, NC 27516, USA.
  {\texttt{drwells@email.unc.edu}}}

\renewcommand{\labelitemi}{--}


\begin{document}
\maketitle

\footnotetext{%
  $^\ast$ This manuscript has been authored by UT-Battelle, LLC under Contract No.
  DE-AC05-00OR22725 with the U.S. Department of Energy.
  % We need to submit the manuscript with the text below. If the editor
  % complains we can remove it.
  The United States
  Government retains and the publisher, by accepting the article for
  publication, acknowledges that the United States Government retains a
  non-exclusive, paid-up, irrevocable, worldwide license to publish or reproduce
  the published form of this manuscript, or allow others to do so, for United
  States Government purposes. The Department of Energy will provide public
  access to these results of federally sponsored research in accordance with the
  DOE Public Access Plan (http://energy.gov/downloads/doe-public-access-plan).
}


\begin{abstract}
  This paper provides an overview of the new features of the finite element
  library \dealii, version 9.8.
\end{abstract}



%%%%%%%%%%%%%%%%%%%%%%%%%%%%%%%%%%%%%%%%%%%%%%%%%%%%%%%%%%%%%%%%%%%%%%%%%%%%%%%%
%%%%%%%%%%%%%%%%%%%%%%%%%%%%%%%%%%%%%%%%%%%%%%%%%%%%%%%%%%%%%%%%%%%%%%%%%%%%%%%%
%%%%%%%%%%%%%%%%%%%%%%%%%%%%%%%%%%%%%%%%%%%%%%%%%%%%%%%%%%%%%%%%%%%%%%%%%%%%%%%%
\section{Overview}

\dealii version 9.7.0 was released August 11, 2024.
\todo{Update}
This paper provides an
overview of the new features of this release and serves as a citable
reference for the \dealii software library version 9.7. \dealii is an
object-oriented finite element library used around the world in the
development of finite element solvers. It is available for free under the
terms of the \emph{GNU Lesser General Public License} (LGPL). The \dealii
project is in the process of relicensing the library under the terms of
the \emph{Apache License 2.0 with LLVM Exception}. Downloads are
available at \url{https://www.dealii.org/} and
\url{https://github.com/dealii/dealii}.

The major changes of this release are:
%
\begin{itemize}
\item
  \todo[inline]{Update list}
\end{itemize}
%

While all of these major changes are discussed in detail in
Section~\ref{sec:major}, there
are a number of other noteworthy changes in the current \dealii release,
which we briefly outline in the remainder of this section:
%
\begin{itemize}
\item
\todo[inline]{Update list}
\end{itemize}
%
The
\href{https://dealii.org/developer/doxygen/deal.II/changes_between_9_5_2_and_9_6_0.html}{changelog}
\todo{Update}
-- listing more than XXX
\todo{Update}
features and bugfixes --
contains a complete record of all changes; see \cite{changes97}.



%%%%%%%%%%%%%%%%%%%%%%%%%%%%%%%%%%%%%%%%%%%%%%%%%%%%%%%%%%%%%%%%%%%%%%%%%%%%%%%%
%%%%%%%%%%%%%%%%%%%%%%%%%%%%%%%%%%%%%%%%%%%%%%%%%%%%%%%%%%%%%%%%%%%%%%%%%%%%%%%%
%%%%%%%%%%%%%%%%%%%%%%%%%%%%%%%%%%%%%%%%%%%%%%%%%%%%%%%%%%%%%%%%%%%%%%%%%%%%%%%%
\section{Major changes to the library}
\label{sec:major}

This release of \dealii contains a number of large and significant changes,
which we will discuss in this section.


\todo[inline]{Add sections}

%%%%%%%%%%%%%%%%%%%%%%%%%%%%%%%%%%%%%%%%%%%%%%%%%%%%%%%%%%%%%%%%%%%%%%%%%%%%%%%%
\subsection{Work towards C++20 modules}
\label{subsec:modules}

\dealii{} currently requires a compiler that understands C++17, but
some additional features are enabled if a compiler supports the C++20
standard. C++20 also introduces ``modules'', a way by which software
packages can export their public interface without relying on the
mechanism of textual inclusion of header files that C++ has inherited from C. The
end goal of C++20 modules is that one can replace the use of long
lists of statements such as\\
\hspace*{1cm}\texttt{\#include <deal.II/grid/tria.h}\\
by a simple\\
\hspace*{1cm}\texttt{import dealii;}

A lot of effort -- perhaps 6 weeks of full-time work, and around 150
pull requests -- has gone into evaluating whether \dealii{} can be
built using modules (in addition to header files, which will need to
be provided for a long time to come for backward compatibility). A
detailed accounting of this effort can be found in 
\cite{bangerth2025experienceconvertinglargemathematical}. The majority
of this work addressed features of the \dealii{} code base,
accumulated over its more than 25 year history, that were acceptable
in a header-based system but not in a module-based system. To name
just one of many examples, there can be cycles of header files that mutually
\texttt{\#include} each other, but this is not allowed for
modules. Most of the pull requests for this project therefore simply
cleaned up our code base, though without having to introduce any
incompatibilities.

As shown in \cite{bangerth2025experienceconvertinglargemathematical},
the use of C++20 modules
reduces compile times for the library itself, though the evidence is
mixed for tutorial programs or applications that build on
\dealii{}. Given that at the moment few systems have compilers, CMake
versions, and build systems that are sufficiently new to support
module builds, it will be several years before this feature will
become widely available. However, the infrastructure for it is now
largely in place.


%%%%%%%%%%%%%%%%%%%%%%%%%%%%%%%%%%%%%%%%%%%%%%%%%%%%%%%%%%%%%%%%%%%%%%%%%%%%%%%%
\subsection{New and improved tutorials and code gallery programs}
\label{subsec:steps}

Many of the \dealii tutorial programs were revised in a variety of ways
as part of this release: Around 130 of the more than 1600 (non-merge)
commits that went into this release touched the tutorial.
% data generated using these commands:
% - tutorial: git log --since 2023/07/07 --until 2024/08/11 --no-merges examples | grep commit | wc -l
% - total:    git log --since 2023/07/07 --until 2024/08/11 --no-merges | grep commit | wc -l
In addition, there are a number of new tutorial
programs:
\begin{itemize}
  \item
    \step{93}
    demonstrates how one can implement problems in which some of the
    variables to be solved for are not degrees of freedom of a finite
    element field (so-called ``non-local'' degrees of freedom, because
    they have no associated node functional that is tied to a cell or
    a patch of cells). Specifically, the program solves an
    optimization problem in which some of the variables are scalars
    that multiply certain source terms.
    \step{93} was written by Sam Scheuerman.
  \item
    \step{97}
    is a program that solves a curl-curl problem from electromagnetics
    -- more specifically, from magnetostatics. The program shows a
    formulation in which two curl-curl problems are solved
    consecutively to model the magnetic field that results from the
    current in an electrically conducting coil.
    \step{97} was written by Siarhei Uzunbajakau.
\end{itemize}

In addition, there is one new program in the code gallery (a collection of
user-contributed programs that often solve more complicated problems
than tutorial programs, and that are intended as starting points for further
research rather than as teaching tools):
\begin{itemize}
  \item \emph{``Phase field facture model in 3D''},
    contributed by Wasim Niyaz Munshi,
    Chandrasekhar Annavarapu,
    Wolfgang Bangerth, and
    Marc Fehling.
\end{itemize}



%%%%%%%%%%%%%%%%%%%%%%%%%%%%%%%%%%%%%%%%%%%%%%%%%%%%%%%%%%%%%%%%%%%%%%%%%%%%%%%%
\subsection{Incompatible changes}\label{subsec:deprecated}

The 9.7 release includes
\href{https://dealii.org/developer/doxygen/deal.II/changes_between_9_5_2_and_9_6_0.html}
     {around 40 incompatible changes};
     \todo{Update link and number}
see \cite{changes97}. Many of these
incompatibilities change internal
interfaces that are not usually used in external
applications. That said, the following are worth mentioning since they
are more broadly visible:
\begin{itemize}
  \item \todo[inline]{Update}
\end{itemize}



%%%%%%%%%%%%%%%%%%%%%%%%%%%%%%%%%%%%%%%%%%%%%%%%%%%%%%%%%%%%%%%%%%%%%%%%%%%%%%%%
%%%%%%%%%%%%%%%%%%%%%%%%%%%%%%%%%%%%%%%%%%%%%%%%%%%%%%%%%%%%%%%%%%%%%%%%%%%%%%%%
%%%%%%%%%%%%%%%%%%%%%%%%%%%%%%%%%%%%%%%%%%%%%%%%%%%%%%%%%%%%%%%%%%%%%%%%%%%%%%%%
\section{How to cite \dealii}\label{sec:cite}

In order to justify the work the developers of \dealii put into this
software, we ask that papers using the library reference one of the
\dealii papers. This helps us justify the effort we put into this library.

There are various ways to reference \dealii. To acknowledge the use of
the current version of the library, \textbf{please reference the present
  document}. For up-to-date information and a bibtex entry
see
\begin{center}
  \url{https://www.dealii.org/publications.html}
\end{center}

The original \dealii paper containing an overview of its
architecture is \cite{BangerthHartmannKanschat2007}, and a more recent
publication documenting \dealii's design decisions is available as \cite{dealII2020design}. If you rely on
specific features of the library, please consider citing any of the
following:
\begin{multicols}{2}
  \vspace*{-36pt}
  \begin{itemize}[leftmargin=4mm]
    \item For geometric multigrid: \cite{Kanschat2004,JanssenKanschat2011,ClevengerHeisterKanschatKronbichler2019, munch2022gc};
    \item For distributed parallel computing: \cite{BangerthBursteddeHeisterKronbichler11};
    \item For $hp$-adaptivity: \cite{BangerthKayserHerold2007,fehlingbangerth2023};
    \item For partition-of-unity (PUM) and finite element enrichment methods:
           \cite{Davydov2016};
    \item For matrix-free and fast assembly techniques:
          \cite{KronbichlerKormann2012,KronbichlerKormann2019};
    \item For computations on lower-dimensional manifolds:
          \cite{DeSimoneHeltaiManigrasso2009};
    \item For curved geometry representations and manifolds:
          \cite{HeltaiBangerthKronbichlerMola2019};
    \item For integration with CAD files and tools:
          \cite{HeltaiMola2015};
    \item For boundary element computations:
          \cite{GiulianiMolaHeltai-2018-a};
    \item For the \texttt{LinearOperator} and
      \texttt{Packaged\-Operation} facilities:
          \cite{MaierBardelloniHeltai-2016-a,MaierBardelloniHeltai-2016-b};
    \item For uses of the \texttt{WorkStream} interface:
          \cite{TKB16};
    \item For uses of the \texttt{ParameterAcceptor} concept, the
          \texttt{MeshWorker::ScratchData} base class, and the
          \texttt{ParsedConvergenceTable} class:
          \cite{SartoriGiulianiBardelloni-2018-a};
    \item For uses of the particle functionality in \dealii:
          \cite{GLHPB18}.
          \vfill\null
  \end{itemize}
\end{multicols}

\dealii can interface with many other libraries:
\todo[inline]{Any new interfaces?}
\todo[inline]{For the references below, if a reference is to a
  website, we should make sure we update that to the current year (and
  perhaps author list).}
\begin{multicols}{3}
  \begin{itemize}[leftmargin=4mm]
    \item ADOL-C \cite{griewank1996adolc}
    \item ArborX \cite{lebrun2020arborx}
    \item ARPACK \cite{lehoucq1998arpack}
    \item Assimp \cite{schulze2021assimp}
    \item BLAS and LAPACK \cite{anderson1999lapack}
    \item Boost \cite{boost-web-page}
    \item CGAL \cite{cgal-user-ref}
    \item cuSOLVER \cite{cusolver-web-page}
    \item cuSPARSE \cite{cusparse-web-page}
    \item Gmsh \cite{geuzaine2009gmsh}
    \item GSL \cite{galassi2009gsl,gsl-web-page}
    \item Ginkgo \cite{anzt2020ginkgo,anzt2022ginkgo}
    \item HDF5 \cite{hdf5-web-page}
    \item METIS \cite{karypis1998metis}
    \item MUMPS \cite{amestoy2001mumps,amestoy2019mumps}
    \item muparser \cite{muparser-web-page}
    \item OpenCASCADE \cite{opencascade-web-page}
    \item p4est \cite{burstedde2011p4est}
    \item PETSc \cite{petsc-user-ref,petsc-web-page}
    \item ROL \cite{ridzal2014rol}
    \item ScaLAPACK \cite{blackford1997scalapack}
    \item SLEPc \cite{hernandez2005slepc}
    \item SUNDIALS \cite{hindmarsh2005sundials}
    \item SymEngine \cite{symengine-web-page}
    \item Taskflow \cite{huang2021taskflow}
    \item TBB \cite{reinders2007tbb}
    \item Trilinos \cite{heroux2005trilinos,trilinos-web-page}
    \item UMFPACK \cite{davis2004umfpack}
  \end{itemize}
\end{multicols}
Please consider citing the appropriate references if you use
interfaces to these libraries.

The two previous releases of \dealii can be cited as
\cite{dealII95,dealII96}.


\section{Acknowledgments}

\dealii is a worldwide project with dozens of contributors around the
globe. Other than the authors of this paper, the following people
contributed code to this release:\\

\todo[inline]{Add list of contributors}
\begin{quote}
\end{quote}
Their contributions are much appreciated!


\bigskip

\dealii and its developers are financially supported through a
variety of funding sources:

\todo[inline]{Everyone please update as appropriate.}
D.~Arndt and B.~Turcksin: Research sponsored by the Laboratory Directed Research and
Development Program of Oak Ridge National Laboratory, managed by UT-Battelle,
LLC, for the U. S. Department of Energy.

W.~Bangerth was partially supported by Award OAC-1835673
as part of the Cyberinfrastructure for Sustained Scientific Innovation (CSSI)
program.

W.~Bangerth, T.~Heister, and R.~Gassm\"{o}ller were partially
supported by the Computational Infrastructure for Geodynamics initiative
(CIG), through the National Science Foundation (NSF) under Award
No.~EAR-2149126 via The University of California -- Davis.

W.~Bangerth was also partially supported by Award EAR-1925595 and OAC-2410847.

B.~Blais was supported by the National Science and Engineering Research Council of Canada (NSERC)  through the RGPIN-2020-04510 Discovery Grant and the MMIAOW Canada Research Level 2 in Computer-Assisted Design and Scale-up of Alternative Energy Vectors for Sustainable Chemical Processes.

M.~Fehling was also partially supported by the ERC-CZ grant LL2105
CONTACT, funded by the Czech Ministry of Education, Youth and Sports.

R.~Gassm\"{o}ller was also partially supported by NSF Awards EAR-1925677
and EAR-2054605.

T.~Heister were also partiallty supported by NSF Awards OAC-2015848 and EAR-1925575.

T.~Heister was also partially supported by NSF OAC-2410848.

L.~Heltai is a member of Gruppo Nazionale per il Calcolo Scientifico (GNCS) of
Istituto Nazionale di Alta Matematica (INdAM). LH was partially supported by
the Italian Ministry of University and Research (MUR), under the grant MUR PRIN
2022 No. 2022WKWZA8 ``Immersed methods for multiscale and multiphysics problems
(IMMEDIATE)'', and acknowledges the MIUR Excellence Department Project awarded
to the Department of Mathematics, University of Pisa, CUP I57G22000700001. 

M.~Kronbichler and P.~Munch were partially supported by the
German Ministry of Education and Research, project
``PDExa: Optimized software methods for solving partial differential
equations on exascale supercomputers'' and the Bayerisches Kompetenznetzwerk
f\"ur Technisch-Wissen\-schaft\-li\-ches Hoch- und H\"ochstleistungsrechnen
(KONWIHR), project ``Fast and scalable finite element algorithms for coupled
multiphysics problems and non-matching grids''.

M.~Kronbichler and L.~Heltai were partially supported by the European Research
Council (ERC) under the European Union's Horizon 2020 research and innovation
programme (call HORIZON-EUROHPC-JU-2023-COE-03, grant agreement No. 101172493
``dealii-X: an Exascale Framework for Digital Twins of the Human Body'').

M.~Maier was partially supported by NSF Award DMS-2045636 and and by the
Air Force Office of Scientific Research under grant/contract number
FA9550-23-1-0007.

P.~Munch acknowledges the funding by the Swedish Research Council (VR) under grant 2021-04620 and the strategic funding from the IT department of Uppsala University.

M.~Schreter-Fleischhacker was supported by the Austrian Science Fund (FWF) Schrödinger Fellowship (project number: J4577) and by the European Research Council
through the ERC Starting Grant ExcelAM (project number: 101117579).

D.~Wells was supported by NSF Award OAC-1931516.

Clemson University is acknowledged for generous allotment of compute
time on the Palmetto cluster.
\todo{Still relevant?}

The authors acknowledge the Texas Advanced Computing Center (TACC) at The University of Texas at Austin for providing HPC resources that have contributed to the research results reported within this paper. \url{http://www.tacc.utexas.edu}
\todo{Still relevant?}

This work used the \textit{Expanse} HPC system at the San Diego Supercomputer Center (SDSC) at UC San Diego through the CIG Science Gateway and Community Codes for the Geodynamics Community MCA08X011 allocation from the Advanced Cyberinfrastructure Coordination Ecosystem: Services \& Support (ACCESS) program, which is supported by National Science Foundation grants \#2138259, \#2138286, \#2138307, \#2137603, and \#2138296. See \cite{Boerner2023}.
\todo{Still relevant?}


\bibliography{paper}{}
\bibliographystyle{abbrv}

\end{document}
