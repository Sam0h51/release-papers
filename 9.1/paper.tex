\documentclass{ansarticle-preprint}
%\usepackage{ucs}
\usepackage[utf8]{inputenc}
\usepackage{amsmath}
%\usepackage{cite}
\usepackage{anslistings}
\usepackage{multicol}
\usepackage{pdfsync}

\usepackage{pgfplots}
\usepackage{pgfplotstable}

\usepackage{fontenc}
\usepackage{graphicx}
\usepackage{xspace}

\usepackage[normalem]{ulem}

\pgfplotsset{compat=1.9}

\newcommand{\specialword}[1]{\texttt{#1}}
\newcommand{\dealii}{{\specialword{deal.II}}\xspace}
\newcommand{\pfrst}{{\specialword{p4est}}\xspace}
\newcommand{\trilinos}{{\specialword{Trilinos}}\xspace}
\newcommand{\aspect}{\specialword{Aspect}\xspace}
\newcommand{\petsc}{\specialword{PETSc}\xspace}
\newcommand{\cmake}{{\specialword{CMake}}\xspace}
\newcommand{\autoconf}{{\specialword{autoconf}}\xspace}

%
% Author list -- please add yourself in both places below (in
%                alphabetical order) if you think that your
%                contributions to the last release warrant this
%

\hypersetup{
  pdfauthor={
    Daniel Arndt,
    Wolfgang Bangerth,
    Denis Davydov,
    Marc Fehling,
    Graham Harper,
    Timo Heister,
    Luca Heltai,
    Martin Kronbichler,
    Ross Maguire Kynch,
    Matthias Maier,
    Jean-Paul Pelteret,
    Bruno Turcksin,
    David Wells
  },
  pdftitle={The deal.II Library, Version 9.1, 2019},
}

\title{The \dealii Library, Version 9.1}

\author[*1]{Daniel Arndt}
\affil[1]{Computational Engineering and Energy Sciences Group,
  Computional Sciences and Engineering Division,
  Oak Ridge National Laboratory, 1 Bethel Valley Rd.,
  TN 37831, USA.
  {\texttt{\{arndtd,turcksinbr\}@ornl.gov}}}

\author[2]{Wolfgang~Bangerth}
\affil[2]{Department of Mathematics, Colorado State University, Fort
  Collins, CO 80523-1874, USA.
  {\texttt{bangerth@colostate.edu},
   \texttt{harper@math.colostate.edu}}}

\author[3]{Denis~Davydov}
\affil[3]{Chair of Applied Mechanics,
  Friedrich-Alexander-Universit\"{a}t Erlangen-N\"{u}rnberg,
  Egerlandstr.\ 5,
  91058 Erlangen, Germany.
  {\texttt{\{denis.davydov,jean-paul.pelteret\}@fau.de}}}

\author[4]{Marc~Fehling}
\affil[4]{Institute for Advanced Simulation,
  Forschungszentrum J\"{u}lich GmbH,
  52425 J\"{u}lich, Germany.
  {\texttt{m.fehling@fz-juelich.de}}}

\author[2]{Graham Harper}

\author[5]{Timo~Heister}
\affil[5]{Mathematical Sciences,
  O-110 Martin Hall,
  Clemson University,
  Clemson, SC 29634, USA
  and Scientific Computing and Imaging Institute,
  72 S Central Campus Drive, Room 3750
  Salt Lake City, UT 84112.
  {\texttt{heister@sci.utah.edu}}}
%
\author[6]{Luca~Heltai}
\affil[6]{SISSA,
  International School for Advanced Studies,
  Via Bonomea 265,
  34136, Trieste, Italy.
  {\texttt{luca.heltai@sissa.it}}}

\author[7]{Martin~Kronbichler}
\affil[7]{Institute for Computational Mechanics,
  Technical University of Munich,
  Boltzmannstr.~15, 85748 Garching, Germany.
  {\texttt{kronbichler@lnm.mw.tum.de}}}

\author[8]{Ross~Maguire~Kynch}
\affil[8]{Renishaw Plc,
  Miskin, CF72 8XY, Wales, UK.
  {\texttt{rkynch@gmail.com}}}

\author[9]{Matthias~Maier}
\affil[9]{Department of Mathematics,
  Texas A\&M University,
  3368 TAMU,
  College Station, TX 77845, USA.
  {\texttt{maier@math.tamu.edu}}}

\author[3]{Jean-Paul~Pelteret}

\author[*1]{Bruno~Turcksin}

\author[10]{David Wells}
\affil[10]{Department of Mathematics, University of North Carolina,
  Chapel Hill, NC 27516, USA.
  {\texttt{drwells@email.unc.edu}}}

\renewcommand{\labelitemi}{--}


\begin{document}
\maketitle

\footnotetext{%
  $^\ast$ This manuscript has been authored by UT-Battelle, LLC under Contract No.
  DE-AC05-00OR22725 with the U.S. Department of Energy. The United States
  Government retains and the publisher, by accepting the article for
  publication, acknowledges that the United States Government retains a
  non-exclusive, paid-up, irrevocable, worldwide license to publish or reproduce
  the published form of this manuscript, or allow others to do so, for United
  States Government purposes. The Department of Energy will provide public
  access to these results of federally sponsored research in accordance with the
  DOE Public Access Plan (http://energy.gov/downloads/doe-public-access-plan).}

\begin{abstract}
  This paper provides an overview of the new features of the finite element
  library \dealii, version 9.1.
\end{abstract}



%%%%%%%%%%%%%%%%%%%%%%%%%%%%%%%%%%%%%%%%%%%%%%%%%%%%%%%%%%%%%%%%%%%%%%%%%%%%%%%%
%%%%%%%%%%%%%%%%%%%%%%%%%%%%%%%%%%%%%%%%%%%%%%%%%%%%%%%%%%%%%%%%%%%%%%%%%%%%%%%%
%%%%%%%%%%%%%%%%%%%%%%%%%%%%%%%%%%%%%%%%%%%%%%%%%%%%%%%%%%%%%%%%%%%%%%%%%%%%%%%%
\section{Overview}

\dealii version 9.1.0 was released May 21, 2019
This paper provides an
overview of the new features of this major release and serves as a citable
reference for the \dealii software library version 9.1. \dealii is an
object-oriented finite element library used around the world in the
development of finite element solvers. It is available for free under the
GNU Lesser General Public License (LGPL). Downloads are available at
\url{https://www.dealii.org/} and \url{https://github.com/dealii/dealii}.

The major changes of this release are:
%
\begin{itemize}
  \item Improved support for automatic differentiation (see
    Section~\ref{subsec:ad}),
  \item Dedicated support for symbolic algebra (see
    Section~\ref{subsec:sd}),
  \item Full support for $hp$~adaptivity in parallel computations (see
    Section~\ref{subsec:hp}),
  \item Interfaces to the HDF5 file format and libraries (see
    Section~\ref{subsec:hdf5}),
  \item Significantly extended GPU support (see Section~\ref{subsec:gpu}),
  \item Parallel geometric multigrid (GMG) improvements (see
    \cite{ClevengerHeisterKanschatKronbichler2019} and
    Section~\ref{subsec:gmg}),
  \item Four new tutorial programs (step-61, step-62, step-63, step-64),
    as well as one new code gallery program (see
    Section~\ref{subsec:steps}).
\end{itemize}
%
The major changes are discussed in detail in Section~\ref{sec:major}. There
are a number of other noteworthy changes in the current \dealii{} release
that we briefly outline in the remainder of this section:
%
\begin{itemize}
\item
  The release contains bug fixes and performance improvements of the
  matrix-free framework and related geometric multigrid solvers. In
  particular, the implementation of the Chebyshev iteration, an often used
  smoother in the matrix-free context, has been revised to reduce vector
  accesses. This speeds up matrix-free multigrid solvers by up to 10-15\%
  on geometries with affine (parallelogram and parallelpiped) cells, and up
  to 5\% on geometries with cells bounded by curved edges and faces.

\item
  Various variants of geometric multigrid solvers and matrix-free
  implementations were run on up to 304,128 MPI ranks during the acceptance
  phase of the SuperMUC-NG supercomputer, verifying the scalability of our
  implementations. Some performance bottlenecks on distributed
  triangulation with more than 100k ranks have been fixed.

\item A new class \texttt{ParsedConvergenceTable} has been introduced
  that greatly simplifies the construction of convergence tables,
  reading the options for the generation of the table from a parameter
  file, and providing methods that, combined with a parameter file,
  allow one to generate convergence tables using one-liners in user
  codes.

\item
  The \texttt{FE\_BernardiRaugel} class implements the non-standard
  Bernardi-Raugel (BR) element that can be used to construct a stable
  velocity-pressure pair for the Stokes equation \cite{BR85}. The BR
  element is an enriched version of the $Q_1^d$ element with added bubble
  functions on each edge (in 2d) or face (in 3d). It addresses the fact
  that the $Q_1^d\times Q_0$ combination is not inf-sup stable (requiring a
  larger velocity space), and that the $Q_2^d\times Q_0$ combination is
  stable but converges with only first-order at the cost of the large
  number of velocity unknowns. The BR space is thus intermediate to the
  $Q_1^d$ and $Q_2^d$ spaces.

  The element is currently only implemented for parallelogram meshes due to
  difficulties associated with the mapping of shape functions: The shape
  functions of the $Q_1^d$ part of the element need to be mapped as
  scalars, as is common for the vector components of the $Q_1^d$ element;
  on the other hand, the vector-valued edge bubble functions need to be
  mapped using the Piola transform as is common for the Raviart-Thomas
  element. \dealii{} does not currently have the ability to use different
  mappings for individual shape functions, though this functionality is
  planned for the next release.

\item
  The \texttt{FE\_NedelecSZ} class is a new implementation of the
  N{\'e}d{\'e}lec element on quadrilaterals and hexahedra. It is based on
  the work of Zaglmayr \cite{Zag06} and overcomes the sign conflict issues
  present in traditional N{\'e}d{\'e}lec elements that arise from the edge
  and face parameterizations used in the basis functions. Therefore, this
  element should provide consistent results for general quadrilateral and
  hexahedral elements for which the relative orientations of edges and
  faces (as seen from all adjacent cells) are often difficult to establish.
  The \texttt{FE\_NedelecSZ} element addresses the sign conflict problem by
  assigning a globally defined orientation to local edges and faces. A
  detailed overview of the implementation of the \texttt{FE\_NedelecSZ}
  element in \dealii{} can be found in \cite{Kynch2017}.

\item All of the elementary geometrical objects of the library (namely
  \texttt{Point<dim>}, \texttt{Segment<dim>}, and
  \texttt{BoundingBox<dim>}) have been augmented with the traits
  needed to comply with \texttt{boost::geometry} concepts. A new
  interface to \texttt{boost::geometry::index::rtree} has been added
  that simplifies the construction of spatial indices based on points,
  bounding boxes, or segments.
\end{itemize}
%
In addition to these changes, the changelog lists more than 200 other
features and bugfixes.




%%%%%%%%%%%%%%%%%%%%%%%%%%%%%%%%%%%%%%%%%%%%%%%%%%%%%%%%%%%%%%%%%%%%%%%%%%%%%%%%
%%%%%%%%%%%%%%%%%%%%%%%%%%%%%%%%%%%%%%%%%%%%%%%%%%%%%%%%%%%%%%%%%%%%%%%%%%%%%%%%
%%%%%%%%%%%%%%%%%%%%%%%%%%%%%%%%%%%%%%%%%%%%%%%%%%%%%%%%%%%%%%%%%%%%%%%%%%%%%%%%
\section{Major changes to the library}
\label{sec:major}

This release of \dealii contains a number of large and significant changes
that will be discussed in this section.

It of course also contains a
vast number of smaller changes and added functionality; the details of these
can be found
\href{https://dealii.org/developer/doxygen/deal.II/changes_between_9_0_1_and_9_1_0.html}{
in the file that lists all changes for this release}, see \cite{changes91}.
(The file is also linked to from the web site of each release as well as
the release announcement.)

%%%%%%%%%%%%%%%%%%%%%%%%%%%%%%%%%%%%%%%%%%%%%%%%%%%%%%%%%%%%%%%%%%%%%%%%%%%%%%%%
\subsection{Improved support for automatic differentiation}
\label{subsec:ad}

In the previous release, numerous classes that are used to assemble linear systems
and right hand sides, as well those used to define constitutive laws, were given
full support for ``white-listed'' automatically differentiable (AD) number types
from the ADOL-C and Sacado libraries. This included the classes that
represent the local contributions of one cell to the global linear
system (i.e., \texttt{FullMatrix} and \texttt{Vector}) as well as the
classes commonly used for the computations at individual quadrature
points (primarily the \texttt{Tensor} and \texttt{SymmetricTensor}
classes of various ranks).
In the current release we have provided a unified interface to these
AD libraries, focussing on two specific use contexts:
\begin{enumerate}
\item The construction and linearization of finite element residuals; and
\item The construction and linearization of constitutive model kinetic variables.
\end{enumerate}

In the first context, the finite element degrees of freedom are considered the
independent variables. From these primitives, the \texttt{EnergyFunctional} helper
class in the namespace \texttt{Differentiation::AD} may be used to compute both the
residual and its linearization by directly defining the contribution to the
(twice differentiated) scalar total energy functional from each cell. Similarly,
the \texttt{ResidualLinearization} class requires the (once-differentiated) finite
element residual to be defined on a per cell basis,  and this contribution is
automatically linearized.

The second context aims directly at constitutive model formulations, and serves to
compute the directional derivatives of components of (multi-field) constitutive laws
with respect to the scalar, vector, tensor and symmetric tensor fields in terms
of which they are parameterized. The \texttt{ScalarFunction} class may be used to
define a scalar function (such as strain energy function) that may be twice
differentiated, while the \texttt{VectorFunction} may be used to define a vector
function (such as a set of kinematic fields) that may be differentiated once.
Since the total derivatives of all components are computed at once, these two helper
classes provide an interface to retrieve each sub-component of the gradient and
Hessian (for a \texttt{ScalarFunction}) or values and Jacobian (for a
\texttt{VectorFunction}).

Although these aforementioned helper classes have been documented with a specific
use in mind, they remain generic and may (with a reinterpretation of the meaning of
the independent and dependent variables) be used for other purposes as well.
Furthermore, through the implementation of \texttt{TapedDrivers} and
\texttt{TapelessDrivers} classes that interface with the active AD library, the
generic helper classes hide library-dependent implementational details and facilitate
switching between the supported libraries and AD number types based on the
user's requirements.

%%%%%%%%%%%%%%%%%%%%%%%%%%%%%%%%%%%%%%%%%%%%%%%%%%%%%%%%%%%%%%%%%%%%%%%%%%%%%%%%
\subsection{Dedicated support for symbolic algebra, including algebra differentiation}
\label{subsec:sd}
To complement the automatic differentiation features in \dealii{}, this release sees
the first step towards integrating and supporting a highly performant computer algebra
system (CAS) via the SymEngine library \cite{symengine-web-page}.
The \texttt{Expression} class in the namespace \texttt{Differentiation::SD} interfaces
to SymEngine and forms the basis of symbolic computations, offering a full set of
overloaded operators and a C++ style interface. This class offers the following
basic functionality:
\begin{itemize}
\item symbolic variable definition,
\item symbolic function definition,
\item expression creation using standard C++ syntax,
\item expression parsing,
\item comparison operations,
\item logical operations,
\item conditional expression construction,
\item differentiation,
\item substitution (partial and complete), and
\item serialization.
\end{itemize}
\dealii{} now also provides an extensive set of math operations, with a syntax mimicking that used
in the C++ standard library. Using the \texttt{Expression} class as a basis, we
have developed a set of functions that can be used to create \dealii{} \texttt{Tensor}s
and \texttt{SymmetricTensor}s of symbolic variables and symbolic functions. This
gives full symbolic tensor algebra support using the pre-existing \texttt{Tensor}
and \texttt{SymmetricTensor} classes and associated functions. We have
also implemented a set of utility functions with the following features:
\begin{itemize}
\item \texttt{differentiate} scalar expressions, tensor and symmetric tensors of expressions
with respect to other scalar expressions, tensor and symmetric tensors of expressions;
\item create symbolic substitution maps;
\item resolve explicit dependencies between expressions; and
\item perform scalar and tensor valued substitution (including conversion from symbolic to
real-valued scalars and tensors).
\end{itemize}

In the next release we expect to implement classes to assist in performing assembly operations
in the same spirit as that which has been done in the \texttt{Differentiation::AD} namespace,
although in a fully symbolic manner.
We will also address performance issues of the \texttt{Expression} class by leveraging
the optimization capabilities of SymEngine, including common subexpression elimination (CSE),
as well as by generating high performance code-paths to evaluate these expressions through the
use of a custom-generated \texttt{std::function} (so-called ``lambda'' optimization) or by
compiling expressions using the LLVM JIT compiler.

%%%%%%%%%%%%%%%%%%%%%%%%%%%%%%%%%%%%%%%%%%%%%%%%%%%%%%%%%%%%%%%%%%%%%%%%%%%%%%%%
\subsection{Full support for $hp$~adaptivity in parallel computations}
\label{subsec:hp}

\dealii{} has had support for $hp$~adaptive methods since around 2005
(documented in \cite{BangerthKayserHerold2007}) and for parallel
computations on distributed meshes since around 2010 (see
\cite{BangerthBursteddeHeisterKronbichler11}), but not for both at the
same time. The challenges to combine these are related to a number of
areas:
\begin{enumerate}
\item Data structures: The data structures necessary to store the
  indices of degrees of freedom are substantially more
  complicated for $hp$~algorithms than for the $h$~adaptive schemes
  that were already implemented. This is because the number of degrees
  of freedom per cell is now no longer constant. Furthermore,
  faces and edges may need to store more than one set of indices if
  the adjacent cells use different polynomial degrees; in the case of
  edges, the number of sets of indices may also be of variable size.

  All of this poses challenges in the parallel context because some of
  the information may not be known, or not be known right away, for
  cells that are not locally owned (i.e., for ghost and ``artificial''
  cells), and for which the data structures stored on different
  processors have to be reconciled.

\item Algorithms: Already for $h$~adaptive meshes, enumerating all
  degrees of freedom uniquely on the global mesh is difficult, as evidenced by
  the complications of the algorithms shown in Section 3.1 of
  \cite{BangerthBursteddeHeisterKronbichler11}, which requires more
  than one page of text and is implemented in many hundreds of lines of
  code.

  These difficulties are even more pronounced when using $hp$~adaptivity.
  The main obstacle is the desire to unify the indices of
  matching degrees of freedom on adjacent cells whenever elements with
  continuous polynomials are used. For example, the edge degree of freedom
  of a $Q_2$~element has to be merged with the middle one of the three
  edge degrees of freedom of a $Q_4$~element on a neighboring cell.
  Section 4.2 of \cite{BangerthKayserHerold2007} discusses a sequential algorithm
  that eliminates one of these degrees of freedom in favor of another,
  but it introduces a ``master'' and a ``slave'' side of the
  interface. This is of no major consequence in sequential
  computations, but is inconvenient in parallel computations if the
  ``master'' side is a ghost cell whose degree of freedom indices are
  not (yet) available while enumerating local degrees of freedom, or
  if the master is an artificial cell whose information will never be
  available on a processor.

  An earlier implementation of the algorithm enumerating degrees of
  freedom, already available in \dealii{} 9.0, simply did not unify
  indices on processor boundaries. However, this makes the total
  number of degrees of freedom dependent on both the partition of the
  mesh and the number of processors available. We have therefore
  re-implemented the algorithm so that the unification does happen
  also on processor boundaries, and will report on the details
  elsewhere.

\item Data transfer patterns: An important algorithm in parallel
  finite element methods is the exchange of information stored on
  cells during mesh repartitioning. This happens, for example, when
  interpolating the solution from one mesh to the next
  adaptively refined mesh; or when adapting the polynomial degrees
  associated with each cell and repartitioning in order to
  balance the computational cost of each processor's partition. When
  using $h$~adaptive methods, the amount of data associated with each
  cell is fixed and the algorithms that implement the data transfer
  are consequently relatively simple. On the other hand, in $hp$~contexts,
  each cell may have a different number of unknowns
  associated with it, and the algorithms that transfer the data are
  substantially more complicated. In order to implement those, we rely
  on recent enhancements of the \pfrst~library (documented in
  \cite{Burstedde2018}) to transfer data of variable size across
  processors. Furthermore, the amount of data associated with each
  cell may be large on cells with higher polynomial degrees, and
  might profit from compression before sending.

\item Balancing computational cost: For $h$~adaptive algorithms, the
  amount of work associated with each cell is essentially the same,
  both during the assembly of linear systems as well as during the
  solver phase. For $hp$~adaptive methods, this is no longer the
  case. Consequently, balancing the cost of work between different
  processors' partitions is no longer as easy as ensuring that every
  processor owns a roughly equal number of cells. Rather, one needs to
  introduce a weighting factor for each cell that describes its
  relative cost compared to some reference. To make things worse, the
  relative cost of assembly on a cell might not match the relative
  cost of the linear solver associated with this cell, leading to
  difficult trade-offs in defining optimal weighting factors. In this
  release, we supplied the basic functionality to attach any amount
  of weighting factors to cells, but users still have to find reasonable
  weights for themselves.
\end{enumerate}

All of these issues have been addressed in the current release and are
available to users. We will report on the details of the algorithms
and their performance in a separate publication.


%%%%%%%%%%%%%%%%%%%%%%%%%%%%%%%%%%%%%%%%%%%%%%%%%%%%%%%%%%%%%%%%%%%%%%%%%%%%%%%%
\subsection{Interfaces to the HDF5 file format and libraries}
\label{subsec:hdf5}

\marginpar{Missing}

%%%%%%%%%%%%%%%%%%%%%%%%%%%%%%%%%%%%%%%%%%%%%%%%%%%%%%%%%%%%%%%%%%%%%%%%%%%%%%%%
\subsection{GPU support via CUDA}
\label{subsec:gpu}

GPU support was significantly extended for the current release:
\begin{itemize}
  \item The \texttt{CUDAWrappers::PreconditionILU} and \texttt{CUDAWrappers::PreconditionIC}
    classes can now be used for preconditioning \texttt{CUDAWrappers::SparseMatrix} objects.
  \item \texttt{LinearAlgebra::distributed::Vector}: the MPI-parallel vector
    class has gained a second template argument \texttt{MemorySpace} which can
    either be \texttt{Host} or \texttt{CUDA}. In the latter case, the data
    resides in GPU memory. By default, the template parameter is
    \texttt{Host} and the behavior is unchanged compared to previous versions.
    When using CUDA, the ghost exchange can be performed either by first copying
    the relevant data to the host, performing MPI communication, and finally
    move the data to the device or, if CUDA-aware MPI is available, by
    performing MPI communication directly between GPUs.
  \item Constrained degrees of freedom: the matrix-free framework now
    supports constrained degrees of freedom. The implementation is based on
    the algorithms described in \cite{ljungkvist2017}. With this addition,
    both Dirichlet boundary conditions and the constraints arising from
    adaptively refined meshes can be imposed within the matrix-free
    framework.
%     \textcolor{red}{%
%     \sout{The only restriction is that for two-dimensional meshes the
%     finite element degree must be odd. There is no such restriction in
%     three dimensions.}
%     \emph{%
%     WB: Can we explain this restriction? It seems odd (pun intended)...
%     BT: No, we cannot. I have spent at least two full weeks trying to
%     understand why it doesn't work, i.e., gives the wrong result, but I
%     don't know. We decided at the time that even with the current
%     limitation this was a useful addition.
%     MM: I suggest that we then simply not mention the limitation and treat
%     it as a known bug. As it is currently worded it can be easily
%     interpreted as a fundamental limitation, i.\,e. ``FE with even degree
%     are not admissible.''}}

  \item MPI matrix-free computations: using \texttt{LinearAlgebra::distributed::Vector}, the
    matrix-free framework can scale to multiple GPUs by taking
    advantage of MPI. Each MPI process can only use one GPU and therefore, if
    multiple GPUs are available in one node, it is necessary to have as many
    ranks as there are GPUs. Using Nvidia Multi-Process Service (MPS), it is also possible
    for multiple processes to use the same GPU. This can be advantageous if the
    amount of work on one rank is not sufficient to fully utilize a GPU.
\end{itemize}

The matrix-free GPU components integrated in \dealii have been compared against
CPUs in \cite{KronbichlerLjungkvist2019}, where the application to geometric
multigrid solvers is discussed.

%%%%%%%%%%%%%%%%%%%%%%%%%%%%%%%%%%%%%%%%%%%%%%%%%%%%%%%%%%%%%%%%%%%%%%%%%%%%%%%%
\subsection{Parallel geometric multigrid improvements}
\label{subsec:gmg}

\textcolor{red}{Missing: performance optimizations, new features, largest
runs now at XYZ (Kronbichler?)}

%%%%%%%%%%%%%%%%%%%%%%%%%%%%%%%%%%%%%%%%%%%%%%%%%%%%%%%%%%%%%%%%%%%%%%%%%%%%%%%%
\subsection{New and improved tutorial and code gallery programs}
\label{subsec:steps}

Many of the \dealii{} tutorial programs were substantially revised as
part of this release. In particular, we have converted many places
that now allow for simpler code through the use of C++11 features such
as range-based for loops and lambda functions.

In addition, there are four new tutorial programs and one new code
gallery program:
\begin{itemize}
\item \texttt{step-61} is a program that implements the ``weak
  Galerkin'' method to solve the Laplace equation. Weak Galerkin
  methods are related to the Hybridized Discontinuous Galerkin method
  in that they introduce degrees of freedom on the interfaces between
  cells, but they do not require the reformulation of the problem as a
  first-order system and instead re-define what the gradient of a
  discontinuous function is.

\item \texttt{step-62} demonstrates the solution of problems related
  to phononic or photonic crystals. Among the techniques shown in this
  program is the solution of complex-valued linear systems, and the
  use of absorbing boundary conditions through the Perfectly Matched
  Layer technique.

\item \texttt{step-63} implements a multigrid preconditioner
  and solver for the advection-diffusion equation, yielding optimal
  complexity.

\item \texttt{step-64} demonstrates the usage of matrix-free methods on Nvidia GPUs.
GPUs are advantageous for these kind of operations because of their superior
hardware characteristics, in particular a higher memory bandwidth than server
CPUs within a given power envelope.

\item The \texttt{MCMC-Laplace} code gallery program is a code useful
  for the forward solution that is used as a building block in
  Bayesian inverse problems, and for sampling the parameter space
  through a Metropolis-Hastings sampler (a kind of Monte Carlo
  Markov Chain method).
\end{itemize}


%%%%%%%%%%%%%%%%%%%%%%%%%%%%%%%%%%%%%%%%%%%%%%%%%%%%%%%%%%%%%%%%%%%%%%%%%%%%%%%%
\subsection{Incompatible changes}

The 9.1 release includes
\href{https://dealii.org/developer/doxygen/deal.II/changes_between_9_0_1_and_9_1_0.html}
     {around 15 incompatible changes}; see \cite{changes91}. The majority of these changes
should not be visible to typical user codes; some remove previously
deprecated classes and functions; and the majority change internal
interfaces that are not usually used in external
applications. However, some are worth mentioning:
\begin{itemize}
\item The \texttt{VectorView} class was removed. We recommend either copying the
      vector subset into a \texttt{Vector} or using a \texttt{BlockVector}.
\item The function \texttt{Subscriptor::subscribe()}, used through the
  \texttt{SmartPointer} class, now requires a pointer to a
      \texttt{std::atomic<bool>} that tracks whether or not the pointer to the
      subscribed-to object is still valid.
\item The \texttt{ConstraintMatrix} class gained a template parameter for scalar
      type and was been renamed \texttt{AffineConstraints}. Several methods that
      take vectors or matrices as arguments,
      such as \texttt{AffineConstraints::distribute\_local\_to\_global()},
      now require that all matrix and vector arguments have matching number
      types.
\item Similarly, the functions \texttt{create\_mass\_matrix} and
      \texttt{create\_boundary\_mass\_matrix} in the \texttt{MatrixCreator}
      namespace no longer
      support matrix and vector objects of different types.
\end{itemize}



%%%%%%%%%%%%%%%%%%%%%%%%%%%%%%%%%%%%%%%%%%%%%%%%%%%%%%%%%%%%%%%%%%%%%%%%%%%%%%%%
%%%%%%%%%%%%%%%%%%%%%%%%%%%%%%%%%%%%%%%%%%%%%%%%%%%%%%%%%%%%%%%%%%%%%%%%%%%%%%%%
%%%%%%%%%%%%%%%%%%%%%%%%%%%%%%%%%%%%%%%%%%%%%%%%%%%%%%%%%%%%%%%%%%%%%%%%%%%%%%%%
\section{How to cite \dealii}\label{sec:cite}

In order to justify the work the developers of \dealii put into this
software, we ask that papers using the library reference one of the
\dealii papers. This helps us justify the effort we put into it.

There are various ways to reference \dealii. To acknowledge the use of
the current version of the library, \textbf{please reference the present
document}. For up to date information and a bibtex entry for this document
see:
\begin{center}
 \url{https://www.dealii.org/publications.html}
\end{center}

The original \texttt{\dealii} paper containing an overview of its
architecture is \cite{BangerthHartmannKanschat2007}. If you rely on
specific features of the library, please consider citing any of the
following:
\begin{itemize}
 \item For geometric multigrid: \cite{Kanschat2004,JanssenKanschat2011,ClevengerHeisterKanschatKronbichler2019};
 \item For distributed parallel computing: \cite{BangerthBursteddeHeisterKronbichler11};
 \item For $hp$~adaptivity: \cite{BangerthKayserHerold2007};
  \item For partition-of-unity (PUM) and enrichment methods of the
    finite element space: \cite{Davydov2016};
 \item For matrix-free and fast assembly techniques:
   \cite{KronbichlerKormann2012,KronbichlerKormann2019};
 \item For computations on lower-dimensional manifolds:
   \cite{DeSimoneHeltaiManigrasso2009};
 \item For integration with CAD files and tools:
   \cite{HeltaiMola2015};
 \item For Boundary Elements Computations:
   \cite{GiulianiMolaHeltai-2018-a};
 \item For \texttt{LinearOperator} and \texttt{PackagedOperation} facilities:
   \cite{MaierBardelloniHeltai-2016-a,MaierBardelloniHeltai-2016-b}.
 \item For uses of the \texttt{WorkStream} interface:
   \cite{TKB16};
   \item For uses of the \texttt{ParameterAcceptor} concept, the
     \texttt{MeshWorker::ScratchData} base class, and the
     \texttt{ParsedConvergenceTable} class: \cite{SartoriGiulianiBardelloni-2018-a}.
\end{itemize}

\dealii can interface with many other libraries:
\begin{multicols}{3}
\begin{itemize}
\item ADOL-C \cite{Griewank1996a,adol-c}
\item ARPACK \cite{arpack}
\item Assimp \cite{assimp}
\item BLAS and LAPACK \cite{lapack}
\item cuSOLVER \cite{cusolver}
\item cuSPARSE \cite{cusparse}
\item Gmsh \cite{geuzaine2009gmsh}
\item GSL \cite{gsl2016}
\item Ginkgo \cite{ginkgo-web-page}
\item HDF5 \cite{hdf5}
\item METIS \cite{karypis1998fast}
\item MUMPS \cite{ADE00,MUMPS:1,MUMPS:2,mumps-web-page}
\item muparser \cite{muparser-web-page}
\item nanoflann \cite{nanoflann}
\item NetCDF \cite{rew1990netcdf}
\item OpenCASCADE \cite{opencascade-web-page}
\item p4est \cite{Burstedde2018,p4est}
\item PETSc \cite{petsc-user-ref,petsc-web-page}
\item ROL \cite{ridzal2014rapid}
\item ScaLAPACK \cite{slug}
\item SLEPc \cite{Hernandez:2005:SSF}
\item SUNDIALS \cite{sundials}
\item SymEngine \cite{symengine-web-page}
\item TBB \cite{Rei07}
\item Trilinos \cite{trilinos,trilinos-web-page}
\item UMFPACK \cite{umfpack}
\end{itemize}
\end{multicols}
Please consider citing the appropriate references if you use interfaces to these
libraries.

The two previous releases of \dealii can be cited as
\cite{dealII85,dealII90}.


\section{Acknowledgments}

\dealii is a world-wide project with dozens of contributors around the
globe. Other than the authors of this paper, the following people
contributed code to this release:\\
% updated 6/05/2019
  Giovanni Alzetta,
  Mathias Anselmann,
  Daniel Appel,
  Alexander Blank,
  Vishal Boddu,
  Benjamin Brands,
  Pi-Yueh Chuang,
  T. Conrad Clevenger,
  Sambit Das,
  Stefano Dominici,
  Nivesh Dommaraju,
  Niklas Fehn,
  Isuru Fernando,
  Andreas Fink,
  Daniel Garcia-Sanchez,
  Rene Gassm{\"o}ller,
  Alexander Grayver,
  Joshua Hanophy,
  Logan Harbour,
  Daniel Jodlbauer,
  Stefan Kaessmair,
  Eldar Khattatov,
  Alexander Knieps,
  Uwe K{\"o}cher,
  Kurt Kremitzki,
  Dustin Kumor,
  Damien Lebrun-Grandie,
  Jonathan Matthews,
  Stefan Meggendorfer,
  Pratik Nayak,
  Lei Qiao,
  Ce Qin,
  Reza Rastak,
  Roland Richter,
  Alberto Sartori,
  Svenja Schoeder,
  Sebastian Stark,
  Antoni Vidal,
  Jiaxin Wang,
  Yuxiang Wang,
  Zhuoran Wang.

Their contributions are much appreciated!


\bigskip

\dealii and its developers are financially supported through a
variety of funding sources:

D.~Arndt and M.~Kronbichler were partially supported by the German
Research Foundation (DFG) under the project ``High-order discontinuous
Galerkin for the exa-scale'' (\mbox{ExaDG}) within the priority program ``Software
for Exascale Computing'' (SPPEXA).

W.~Bangerth and T.~Heister were partially
supported by the National Science Foundation under award OAC-1835673
as part of the Cyberinfrastructure for Sustained Scientific Innovation (CSSI)
program; by award DMS-1821210; and by the Computational Infrastructure
in Geodynamics initiative (CIG), through the National Science
Foundation under Award No.~EAR-1550901 and The
University of California -- Davis.

D.~Davydov was supported by the German Research Foundation (DFG), grant DA
1664/2-1 and the Bayerisches Kompetenznetzwerk
f\"ur Technisch-Wissenschaftliches Hoch- und H\"ochstleistungsrechnen
(KONWIHR).

T.~Heister was also partially supported by NSF Award DMS-1522191, and
by Technical Data Analysis, Inc. through US Navy SBIR N16A-T003.

M.~Maier was partially supported by ARO MURI Award No. W911NF-14-0247.

B.~Turcksin: Research sponsored by the Laboratory Directed Research and
Development Program of Oak Ridge National Laboratory, managed by UT-Battelle,
LLC, for the U. S. Department of Energy.

D.~Wells was supported by the National Science Foundation (NSF) through Grant
DMS-1344962.

The Interdisciplinary Center for Scientific Computing (IWR) at Heidelberg
University has provided hosting services for the \dealii web page.


\bibliography{paper}{}
\bibliographystyle{abbrv}

\end{document}
