\documentclass{ansarticle-preprint}
%\usepackage{ucs}
\usepackage[utf8]{inputenc}
\usepackage{amsmath}
%\usepackage{cite}
\usepackage{anslistings}
\usepackage{multicol}
\usepackage{pdfsync}
\usepackage{enumitem}

\usepackage{pgfplots}
\usepackage{pgfplotstable}

\usepackage{fontenc}
\usepackage{graphicx}
\usepackage{xspace}

\usepackage{siunitx}

\usepackage{floatflt}

\usepackage{multirow}

\usepackage{booktabs}

%\renewcommand{\baselinestretch}{2.0}
%\usepackage{lineno}
%\renewcommand\linenumberfont{\normalfont\tiny}
%\linenumbers

\graphicspath{{svg/}}

\usepackage[normalem]{ulem}

\usepackage{caption}
\usepackage{subcaption}

\usepackage{todonotes}

\pgfplotsset{compat=1.9}
\definecolor{gnuplot@lightblue}{RGB}{87,181,232}
\definecolor{gnuplot@green}{RGB}{0,158,115}
\definecolor{gnuplot@purple}{RGB}{148,0,212}

\newcommand{\specialword}[1]{\texttt{#1}}
\newcommand{\dealii}{{\specialword{deal.II}}\xspace}
\newcommand{\pfrst}{{\specialword{p4est}}\xspace}
\newcommand{\trilinos}{{\specialword{Trilinos}}\xspace}
\newcommand{\aspect}{\specialword{Aspect}\xspace}
\newcommand{\petsc}{\specialword{PETSc}\xspace}
\newcommand{\snes}{{\specialword{SNES}}\xspace}
\newcommand{\ts}{{\specialword{TS}}\xspace}
\newcommand{\petscsf}{{\specialword{SF}}\xspace}
\newcommand{\cmake}{{\specialword{CMake}}\xspace}
\newcommand{\candi}{{\specialword{candi}}\xspace}
\newcommand{\sundials}{{\specialword{SUNDIALS}}\xspace}
\newcommand{\kinsol}{{\specialword{KINSOL}}\xspace}
\newcommand{\ida}{{\specialword{IDA}}\xspace}
\newcommand{\arkode}{{\specialword{ARKODE}}\xspace}
\newcommand{\boost}{{\specialword{Boost}}\xspace}
\newcommand{\kokkos}{{\specialword{Kokkos}}\xspace}


\usetikzlibrary{shapes.misc}
\tikzset{cross/.style={cross out, draw=black, minimum size=2*(#1-\pgflinewidth), inner sep=0pt, outer sep=0pt},
%default radius will be 1pt.
cross/.default={2pt}}

%
% Author list -- please add yourself in both places below (in
%                alphabetical order) if you think that your
%                contributions to the last release warrant this
%

\hypersetup{
  pdfauthor={
    Pasquale Claudio Africa,
    Daniel Arndt,
    Wolfgang Bangerth,
    Bruno Blais,
    Marc Fehling,
    Rene Gassm"{o}ller,
    Timo Heister,
    Luca Heltai,
    Martin Kronbichler,
    Matthias Maier,
    Peter Munch,
    Magdalena Schreter-Fleischhacker,
    Jan Philipp Thiele,
    Bruno Turcksin,
    David Wells,
    Vladimir Yushutin
  },
  pdftitle={The deal.II Library, Version 9.6, 2024},
}

\title{The \dealii Library, Version 9.6}

 \author[1*]{Pasquale~Claudio~Africa}
 \affil[1]{SISSA International School for Advanced Studies,
   mathLab,
   Via Bonomea, 265,
   34136O, Trieste, Italy.
   \texttt{pafrica@sissa.it}}

 \author[2*]{Daniel Arndt}
 \affil[2]{Computational Coupled Physics Group,
   Computational Sciences and Engineering Division,
   Oak Ridge National Laboratory, 1 Bethel Valley Rd.,
   TN 37831, USA.
   \texttt{arndtd/turcksinbr@ornl.gov}}

 \author[3,4]{Wolfgang~Bangerth}
 \affil[3]{Department of Mathematics, Colorado State University, Fort
   Collins, CO 80523-1874, USA.
   \texttt{bangerth@colostate.edu}}
 \affil[4]{Department of Geosciences, Colorado State University, Fort
   Collins, CO 80523, USA.}

\author[5]{Bruno Blais}
\affil[5]{Chemical Engineering High-performance Analysis, Optimization and Simulation (CHAOS) laboratory, Department of Chemical Engineering,
             Polytechnique Montréal,
             PO Box 6079, Stn Centre-Ville, Montréal, Québec, Canada, H3C 3A7.
             {\texttt{bruno.blais@polymtl.ca}}}

\author[6]{Marc~Fehling}
 \affil[6]{Department of Mathematical Analysis,
    Faculty of Mathematics and Physics, Charles University,
    Sokolovsk{\'a} 49/83, 186\,75 Prague 8, Czech Republic.
    {\texttt{marc.fehling@matfyz.cuni.cz}}}


\author[7]{Rene~Gassm"{o}ller}
\affil[7]{GEOMAR Helmholtz Centre for Ocean Research Kiel, 24148 Kiel, Germany}

\author[8]{Timo~Heister}
 \affil[8]{School of Mathematical and Statistical Sciences,
   Clemson University,
   Clemson, SC, 29634, USA.
   {\texttt{heister@clemson.edu}}}

\author[9]{Luca~Heltai}
\affil[9]{University of Pisa, Italy.}

 \author[10,11]{Martin~Kronbichler}
 \affil[10]{Faculty of Mathematics, Ruhr University Bochum,
   Universit\"atsstr.~150, 44780 Bochum, Germany.
 {\texttt{martin.kronbichler@rub.de}}}
 \affil[11]{Institute of Mathematics,
   University of Augsburg,
   Universit\"atsstr.~12a, 86159 Augsburg, Germany.
   }

\author[12]{Matthias~Maier}
\affil[12]{Department of Mathematics,
  Texas A\&M University,
  3368 TAMU,
  College Station, TX 77845, USA.
  {\texttt{maier@math.tamu.edu}}}

\author[11,13]{Peter Munch}
 \affil[13]{Uppsala University, Sweden.
  {\texttt{peter.munch@it.uu.se}}}

\author[14]{Magdalena Schreter-Fleischhacker}
 \affil[14]{Institute for Computational Mechanics, Technical University of Munich, Boltzmannstraße 15, 85748 Garching, Germany.
  {\texttt{magdalena.schreter@tum.de}}}

\author[15]{Jan Philipp Thiele}
 \affil[15]{Weierstrass Institute for Applied Analysis and Stochastics,\newline
 Leibniz Institute in Forschungsverbund Berlin e.V.
  {\texttt{thiele@wias-berlin.de}}}

\author[2]{Bruno~Turcksin}

\author[16]{David Wells}
\affil[16]{Department of Mathematics, University of North Carolina,
  Chapel Hill, NC 27516, USA.
  {\texttt{drwells@email.unc.edu}}}

\author[17,8]{Vladimir Yushutin}
\affil[17]{Department of Mathematics, University of Tennessee at Knoxville,
	 Knoxville TN 37996-1320, USA.
{\texttt{vyushuti@utk.edu}}}


\renewcommand{\labelitemi}{--}


\begin{document}
\maketitle

\footnotetext{%
  $^\ast$ This manuscript has been authored by UT-Battelle, LLC under Contract No.
  DE-AC05-00OR22725 with the U.S. Department of Energy.
  % We need to submit the manuscript with the text below. If the editor
  % complains we can remove it.
  The United States
  Government retains and the publisher, by accepting the article for
  publication, acknowledges that the United States Government retains a
  non-exclusive, paid-up, irrevocable, worldwide license to publish or reproduce
  the published form of this manuscript, or allow others to do so, for United
  States Government purposes. The Department of Energy will provide public
  access to these results of federally sponsored research in accordance with the
  DOE Public Access Plan (http://energy.gov/downloads/doe-public-access-plan).
}


\begin{abstract}
  This paper provides an overview of the new features of the finite element
  library \dealii, version 9.6.
\end{abstract}



%%%%%%%%%%%%%%%%%%%%%%%%%%%%%%%%%%%%%%%%%%%%%%%%%%%%%%%%%%%%%%%%%%%%%%%%%%%%%%%%
%%%%%%%%%%%%%%%%%%%%%%%%%%%%%%%%%%%%%%%%%%%%%%%%%%%%%%%%%%%%%%%%%%%%%%%%%%%%%%%%
%%%%%%%%%%%%%%%%%%%%%%%%%%%%%%%%%%%%%%%%%%%%%%%%%%%%%%%%%%%%%%%%%%%%%%%%%%%%%%%%
\section{Overview}

\dealii version 9.6.0 was released X X, 2024.
This paper provides an
overview of the new features of this release and serves as a citable
reference for the \dealii software library version 9.6. \dealii is an
object-oriented finite element library used around the world in the
development of finite element solvers. It is available for free under the
GNU Lesser General Public License (LGPL). Downloads are available at
\url{https://www.dealii.org/} and \url{https://github.com/dealii/dealii}.

The major changes of this release are:
%
\begin{itemize}
  \item 
\end{itemize}
%

While all of these major changes are discussed in detail in
Section~\ref{sec:major}, there
are a number of other noteworthy changes in the current \dealii release,
which we briefly outline in the remainder of this section:
%
\begin{itemize}
\item changes of interface of AffineConstraints
\item clean up: orientation
\item various performance improvements throughout the library
\item the \texttt{SolverGMRES} class now offers a third orthogonalization
  method, the classical Gram--Schmidt method with delayed orthogonalization
  \cite{Bielich2022}. Furthermore, the solver now specifies the maximal basis
  size of the Arnoldi basis, rather than the number of auxiliary vectors.
\end{itemize}
%
The changelog lists more than X other features and bugfixes.


%%%%%%%%%%%%%%%%%%%%%%%%%%%%%%%%%%%%%%%%%%%%%%%%%%%%%%%%%%%%%%%%%%%%%%%%%%%%%%%%
%%%%%%%%%%%%%%%%%%%%%%%%%%%%%%%%%%%%%%%%%%%%%%%%%%%%%%%%%%%%%%%%%%%%%%%%%%%%%%%%
%%%%%%%%%%%%%%%%%%%%%%%%%%%%%%%%%%%%%%%%%%%%%%%%%%%%%%%%%%%%%%%%%%%%%%%%%%%%%%%%
\section{Major changes to the library}
\label{sec:major}

This release of \dealii contains a number of large and significant changes,
which will be discussed in this section.
It of course also includes a
vast number of smaller changes and added functionality; the details of these
can be found
\href{https://dealii.org/developer/doxygen/deal.II/changes_between_9_4_2_and_9_5_0.html}
{in the file that lists all changes for this release}; see \cite{changes95}.


\begin{itemize}
\item Hermite element
\item simplices: cubic simplex elements, improved refinement of tetrahedron cells
\item Tpetra advances
\item Nedelec for meshes with hanging nodes
\end{itemize}


%%%%%%%%%%%%%%%%%%%%%%%%%%%%%%%%%%%%%%%%%%%%%%%%%%%%%%%%%%%%%%%%%%%%%%%%%%%%%%%%
\subsection{Relicensing to Apache 2.0}\label{sec:license}



%%%%%%%%%%%%%%%%%%%%%%%%%%%%%%%%%%%%%%%%%%%%%%%%%%%%%%%%%%%%%%%%%%%%%%%%%%%%%%%%
\subsection{Updates to matrix-free algorithms}\label{sec:mf}

\begin{itemize}
\item VectorizedArray for ARM Neon
\item Improvement for evaluation of values and gradients for
  $H$(div)-conforming Raviart--Thomas elements on non-Cartesian elements
\item Improved internal data structures of the tensor-product evaluators as
  well as the evaluators for simplex elements, which speed up the evaluation
  in several scenarios, especially multi-component systems.
\end{itemize}

%%%%%%%%%%%%%%%%%%%%%%%%%%%%%%%%%%%%%%%%%%%%%%%%%%%%%%%%%%%%%%%%%%%%%%%%%%%%%%%%
\subsection{Advances in non-matching support}\label{sec:nonmatching}

\begin{itemize}
\item non-nested multigrid: new features and performance improvement
\item FERemoteEvaluation
\item FECouplingValues
\end{itemize}


%%%%%%%%%%%%%%%%%%%%%%%%%%%%%%%%%%%%%%%%%%%%%%%%%%%%%%%%%%%%%%%%%%%%%%%%%%%%%%%%
\subsection{New and improved tutorials and code gallery programs}
\label{subsec:steps}

Many of the \dealii tutorial programs were revised in a variety of ways
as part of this release. In addition, there are a number of new tutorial
programs:
\begin{itemize}
  \item
    \texttt{step-83} 
    demonstrates how one can implement
    checkpoint/restart functionality in \dealii{}-based programs,
    using the BOOST serialization functionality as a
    foundation. step-83 was written by Pasquale Africa, Wolfgang
    Bangerth, and Bruno Blais and uses step-19 as its basis.
  \item
    \texttt{step-86}
    is a program that solves the heat equation using PETSc's TS (time
    stepping) framework for the solution of ordinary differential
    equations. Written by Wolfgang Bangerth (Colorado State
    University), Luca Heltai (University of Pisa), and Stefano Zampini
    (King Abdullah University of Science and Technology), it
    illustrates how PDE solvers for time-dependent problems can be
    integrated with existing ODE solver packages to use advanced ODE
    solver concepts (such as higher-order time integration methods and
    adaptive time step control), all without sacrificing the things
    that have traditionally led code authors toward writing their own
    time stepping routines (such as wanting to change the mesh every
    once in a while, or having to deal with boundary conditions).
  \item
    \texttt{step-87} was contributed by Magdalena Schreter-Fleischhacker
    (Technical University of Munich) and Peter Munch
    (University of Augsburg/Uppsala University). It
    presents the advanced point-evaluation functionalities of \dealii
    that are useful for evaluating finite element solutions at
    arbitrary points on finite element meshes that can be distributed among processes.
  \item
    \texttt{step-89} was contributed by Johannes Heinz (TU Wien),
    Maximilian Bergbauer (Technical University of Munich),
    Marco Feder (SISSA), and Peter Munch (University of Augsburg/Uppsala University).
    It shows one way how to apply non-matching and/or Chimera methods
    within matrix-free loops in \dealii.
  \item
    \texttt{step-90} was contributed by Vladimir Yushutin and Timo Heister (Clemson University).
    It implements the trace finite element method (TraceFEM). TraceFEM solves PDEs
    posed on a, possibly evolving, $(dim-1)$-dimensional surface $\Gamma$ employing
    a fixed uniform background mesh of a $dim$-dimensional domain in which
    the surface is embedded. Such surface PDEs arise in problems involving
    material films with complex properties and in other situations in which
    a non-trivial condition is imposed on either a stationary or a moving interface.
    The program considers a steady, complex, non-trivial surface and the prototypical
    Laplace-Beltrami equation which is a counterpart of
    the Poisson problem on flat domains.
\end{itemize}

In addition, there are three new programs in the code gallery (a collection of
user-contributed programs that often solve more complicated problems
than tutorial programs, and that are intended as starting points for further
research rather than as teaching tools):
UPDATE
\begin{itemize}
  \item \textit{``Crystal growth phase field model'''},
    contributed by Umair Hussain;
  \item \textit{``Nonlinear heat transfer problem''}, contributed by
    Narasimhan Swaminathan;
  \item \textit{``Traveling-wave solutions of a qualitative model for combustion waves''}, contributed by
    Shamil Magomedov.
\end{itemize}


%%%%%%%%%%%%%%%%%%%%%%%%%%%%%%%%%%%%%%%%%%%%%%%%%%%%%%%%%%%%%%%%%%%%%%%%%%%%%%%%
\subsection{Incompatible changes}\label{subsec:deprecated}

The 9.6 release includes
\href{https://dealii.org/developer/doxygen/deal.II/changes_between_9_5_2_and_9_6_0.html}
     {around X incompatible changes};
see \cite{changes96}. Many of these
incompatibilities change internal
interfaces that are not usually used in external
applications. That said, the following are worth mentioning since they
may have been more widely used:
\begin{itemize}
  \item CUDAWrappers
\end{itemize}



%%%%%%%%%%%%%%%%%%%%%%%%%%%%%%%%%%%%%%%%%%%%%%%%%%%%%%%%%%%%%%%%%%%%%%%%%%%%%%%%
%%%%%%%%%%%%%%%%%%%%%%%%%%%%%%%%%%%%%%%%%%%%%%%%%%%%%%%%%%%%%%%%%%%%%%%%%%%%%%%%
%%%%%%%%%%%%%%%%%%%%%%%%%%%%%%%%%%%%%%%%%%%%%%%%%%%%%%%%%%%%%%%%%%%%%%%%%%%%%%%%
\section{How to cite \dealii}\label{sec:cite}

In order to justify the work the developers of \dealii put into this
software, we ask that papers using the library reference one of the
\dealii papers. This helps us justify the effort we put into this library.

There are various ways to reference \dealii. To acknowledge the use of
the current version of the library, \textbf{please reference the present
  document}. For up-to-date information and a bibtex entry
see
\begin{center}
  \url{https://www.dealii.org/publications.html}
\end{center}

The original \dealii paper containing an overview of its
architecture is \cite{BangerthHartmannKanschat2007}, and a more recent
publication documenting \dealii's design decisions is available as \cite{dealII2020design}. If you rely on
specific features of the library, please consider citing any of the
following:
\begin{multicols}{2}
  \vspace*{-36pt}
  \begin{itemize}[leftmargin=4mm]
    \item For geometric multigrid: \cite{Kanschat2004,JanssenKanschat2011,ClevengerHeisterKanschatKronbichler2019, munch2022gc};
    \item For distributed parallel computing: \cite{BangerthBursteddeHeisterKronbichler11};
    \item For $hp$-adaptivity: \cite{BangerthKayserHerold2007,fehlingbangerth2023};
    \item For partition-of-unity (PUM) and finite element enrichment methods:
           \cite{Davydov2016};
    \item For matrix-free and fast assembly techniques:
          \cite{KronbichlerKormann2012,KronbichlerKormann2019};
    \item For computations on lower-dimensional manifolds:
          \cite{DeSimoneHeltaiManigrasso2009};
    \item For curved geometry representations and manifolds:
          \cite{HeltaiBangerthKronbichlerMola2019};
    \item For integration with CAD files and tools:
          \cite{HeltaiMola2015};
    \item For boundary element computations:
          \cite{GiulianiMolaHeltai-2018-a};
    \item For the \texttt{LinearOperator} and
      \texttt{Packaged\-Operation} facilities:
          \cite{MaierBardelloniHeltai-2016-a,MaierBardelloniHeltai-2016-b};
    \item For uses of the \texttt{WorkStream} interface:
          \cite{TKB16};
    \item For uses of the \texttt{ParameterAcceptor} concept, the
          \texttt{MeshWorker::ScratchData} base class, and the
          \texttt{ParsedConvergenceTable} class:
          \cite{SartoriGiulianiBardelloni-2018-a};
    \item For uses of the particle functionality in \dealii:
          \cite{GLHPB18}.
          \vfill\null
  \end{itemize}
\end{multicols}

\dealii can interface with many other libraries:
\begin{multicols}{3}
  \begin{itemize}[leftmargin=4mm]
    \item ADOL-C \cite{griewank1996adolc}
    \item ArborX \cite{lebrun2020arborx}
    \item ARPACK \cite{lehoucq1998arpack}
    \item Assimp \cite{schulze2021assimp}
    \item BLAS and LAPACK \cite{anderson1999lapack}
    \item Boost \cite{boost-web-page}
    \item CGAL \cite{cgal-user-ref}
    \item cuSOLVER \cite{cusolver-web-page}
    \item cuSPARSE \cite{cusparse-web-page}
    \item Gmsh \cite{geuzaine2009gmsh}
    \item GSL \cite{galassi2009gsl,gsl-web-page}
    \item Ginkgo \cite{anzt2020ginkgo,anzt2022ginkgo}
    \item HDF5 \cite{hdf5-web-page}
    \item METIS \cite{karypis1998metis}
    \item MUMPS \cite{amestoy2001mumps,amestoy2019mumps}
    \item muparser \cite{muparser-web-page}
    \item OpenCASCADE \cite{opencascade-web-page}
    \item p4est \cite{burstedde2011p4est}
    \item PETSc \cite{petsc-user-ref,petsc-web-page}
    \item ROL \cite{ridzal2014rol}
    \item ScaLAPACK \cite{blackford1997scalapack}
    \item SLEPc \cite{hernandez2005slepc}
    \item SUNDIALS \cite{hindmarsh2005sundials}
    \item SymEngine \cite{symengine-web-page}
    \item TBB \cite{reinders2007tbb}
    \item Trilinos \cite{heroux2005trilinos,trilinos-web-page}
    \item UMFPACK \cite{davis2004umfpack}
  \end{itemize}
\end{multicols}
Please consider citing the appropriate references if you use
interfaces to these libraries.

The two previous releases of \dealii can be cited as
\cite{dealII92,dealII93}.


\section{Acknowledgments}

\dealii is a worldwide project with dozens of contributors around the
globe. Other than the authors of this paper, the following people
contributed code to this release:\\
%
% Uwe Koecher doesn't usually show up in the changelog, but
% we should make sure he's listed.
%

% This is up-to-date as of 2024-07-23 11:52PM CEST
\begin{quote}
Laryssa      Abdala,
Mathias      Anselmann,
Abbas        Ballout,
Maximilian   Bergbauer,
Julian       Brotz,
Marco        Feder,
Niklas       Fehn,
Menno        Fraters,
Quang        Hoang,
Vladimir     Ivannikov,
Tao          Jin,
Yimin        Jin,
Sebastian    Kinnewig,
Paras        Kumar,
Sébastien    Loriot,
Nils         Much,
Abdullah     Mujahid,
Bob          Myhill,
Paul A.      Patience,
Luz          Paz,
Laura        Prieto Saavedra,
Sebastian    Proell,
Hendrik      Ranocha,
Johannes     Resch,
Andreas      Ritthaler,
Malik        Scheifinger,
David        Schneider,
Magdalena    Schreter-Fleischhacker,
Richard      Schussnig,
Nils         Schween,
Kyle         Schwiebert,
Simranjeet   Singh,
Simon        Sticko,
Dominik      Still,
Jan Philipp  Thiele,
Thierry      Thomas,
Vinayak      Vijay,
Ivy          Weber,
Simon        Wiesheier,
Chengjiang   Yin,
Vladimir     Yushutin,
Stefano      Zampini.
\end{quote}
Their contributions are much appreciated!


\bigskip

\dealii and its developers are financially supported through a
variety of funding sources:

P.C.~Africa was partially supported by the consortium iNEST (Interconnected North-East Innovation Ecosystem),
Piano Nazionale di Ripresa e Resilienza (PNRR) - Missione 4 Componente 2, Investimento 1.5 - D.D. 1058 23/06/2022,
ECS00000043, supported by the European Union's NextGenerationEU program.

D.~Arndt and B.~Turcksin: Research sponsored by the Laboratory Directed Research and
Development Program of Oak Ridge National Laboratory, managed by UT-Battelle,
LLC, for the U. S. Department of Energy.

W.~Bangerth and T.~Heister were partially
supported by the Computational Infrastructure for Geodynamics initiative
(CIG), through the National Science Foundation (NSF) under Award
No.~EAR-1550901 and EAR-2149126 via The University of California -- Davis.

W.~Bangerth and M.~Fehling were partially supported by Award OAC-1835673
as part of the Cyberinfrastructure for Sustained Scientific Innovation (CSSI)
program.

W.~Bangerth was also partially supported by Awards DMS-1821210 and EAR-1925595.

M.~Fehling was also partially supported by the ERC-CZ grant LL2105
CONTACT funded by the Czech Ministry of Education, Youth and Sports.

M.~Bergbauer was supported by the German Research Foundation (DFG) under the
project ``High-Performance Cut Discontinuous Galerkin Methods for Flow
Problems and Surface-Coupled Multiphysics Problems'' Grant Agreement
No.~456365667.

J.~Heinz was supported by the European Union’s Framework Programme for Research
and Innovation Horizon 2020 (2014-2020) under the Marie Sk\l{}odowska--Curie Grant
Agreement No. [812719].

T.~Heister was also partially supported by NSF
Awards OAC-2015848, DMS-2028346, and
EAR-1925575.

L.~Heltai and M~.Feder were partially supported by the Italian Ministry of
University and Research (MUR), under the grant MUR PRIN 2022 No. 2022WKWZA8
``Immersed methods for multiscale and multiphysics problems (IMMEDIATE)''.

M.~Kronbichler and P.~Munch were partially supported by the
German Ministry of Education and Research, project
``PDExa: Optimized software methods for solving partial differential
equations on exascale supercomputers'' and the Bayerisches Kompetenznetzwerk
f\"ur Technisch-Wissen\-schaft\-li\-ches Hoch- und H\"ochstleistungsrechnen
(KONWIHR), projects ``High-order matrix-free finite
element implementations with hybrid parallelization and improved data
locality'' and ``Fast and scalable finite element algorithms for coupled
multiphysics problems and non-matching grids''.

M.~Maier was partially supported by NSF Award DMS-2045636 and and by the
Air Force Office of Scientific Research under grant/contract number
FA9550-23-1-0007.

D.~Wells was supported by the NSF Award OAC-1931516.

S. Zampini was supported by the KAUST Extreme Computing Research Center.

B.~Blais was supported by the National Science and Engineering Research Council of Canada (NSERC)  through the RGPIN-2020-04510 Discovery Grant and the MMIAOW Canada Research Level 2 in Computer-Assisted Design and Scale-up of Alternative Energy Vectors for Sustainable Chemical Processes

Clemson University is acknowledged for generous allotment of compute time on Palmetto cluster.

M.~Schreter-Fleischhacker was supported by the Austrian Science Fund (FWF) Schrödinger Fellowship (project number: J4577) and by the European Research Council
through the ERC Starting Grant ExcelAM (project number: 101117579).

The authors acknowledge the Texas Advanced Computing Center (TACC) at The University of Texas at Austin for providing HPC resources that have contributed to the research results reported within this paper. \url{http://www.tacc.utexas.edu}

\todo[inline]{Update the following to ACCESS, see https://access-ci.org/about/acknowledging-access/}

This work used the Extreme Science and Engineering Discovery Environment (XSEDE), which is supported by National Science Foundation grant number ACI-1053575 access through the CIG Science Gateway and Community Codes for the Geodynamics Community MCA08X011 allocation.


\bibliography{paper}{}
\bibliographystyle{abbrv}

\end{document}
