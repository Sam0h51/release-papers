\documentclass{ansarticle-preprint}
%\usepackage{ucs}
\usepackage[utf8]{inputenc}
\usepackage{amsmath}
%\usepackage{cite}
\usepackage{anslistings}
\usepackage{multicol}
\usepackage{pdfsync}
\usepackage{enumitem}

\usepackage{pgfplots}
\usepackage{pgfplotstable}

\usepackage{fontenc}
\usepackage{graphicx}
\usepackage{xspace}

\usepackage{siunitx}

\usepackage{floatflt}

\usepackage{multirow}

\usepackage{booktabs}

%\renewcommand{\baselinestretch}{2.0}
%\usepackage{lineno}
%\renewcommand\linenumberfont{\normalfont\tiny}
%\linenumbers

\graphicspath{{svg/}}

\usepackage[normalem]{ulem}

\usepackage{caption}
\usepackage{subcaption}

\usepackage{todonotes}

\pgfplotsset{compat=1.9}
\definecolor{gnuplot@lightblue}{RGB}{87,181,232}
\definecolor{gnuplot@green}{RGB}{0,158,115}
\definecolor{gnuplot@purple}{RGB}{148,0,212}

\newcommand{\specialword}[1]{\texttt{#1}}
\newcommand{\dealii}{{\specialword{deal.II}}\xspace}
\newcommand{\pfrst}{{\specialword{p4est}}\xspace}
\newcommand{\trilinos}{{\specialword{Trilinos}}\xspace}
\newcommand{\aspect}{\specialword{Aspect}\xspace}
\newcommand{\petsc}{\specialword{PETSc}\xspace}
\newcommand{\snes}{{\specialword{SNES}}\xspace}
\newcommand{\ts}{{\specialword{TS}}\xspace}
\newcommand{\petscsf}{{\specialword{SF}}\xspace}
\newcommand{\cmake}{{\specialword{CMake}}\xspace}
\newcommand{\candi}{{\specialword{candi}}\xspace}
\newcommand{\MPI}{{\specialword{MPI}}\xspace}
\newcommand{\MPIx}{{\specialword{MPI+X}}\xspace}
\newcommand{\sundials}{{\specialword{SUNDIALS}}\xspace}
\newcommand{\kinsol}{{\specialword{KINSOL}}\xspace}
\newcommand{\ida}{{\specialword{IDA}}\xspace}
\newcommand{\arkode}{{\specialword{ARKODE}}\xspace}
\newcommand{\boost}{{\specialword{Boost}}\xspace}
\newcommand{\kokkos}{{\specialword{Kokkos}}\xspace}

%Trilinos Packages
\newcommand{\epetra}{{\specialword{Epetra}}\xspace}
\newcommand{\petra}{{\specialword{Petra}}\xspace}
\newcommand{\teuchos}{{\specialword{Teuchos}}\xspace}
\newcommand{\tpetra}{{\specialword{Tpetra}}\xspace}


\usetikzlibrary{shapes.misc}
\tikzset{cross/.style={cross out, draw=black, minimum size=2*(#1-\pgflinewidth), inner sep=0pt, outer sep=0pt},
%default radius will be 1pt.
cross/.default={2pt}}

%
% Author list -- please add yourself in both places below (in
%                alphabetical order) if you think that your
%                contributions to the last release warrant this
%

\hypersetup{
  pdfauthor={
    Pasquale Claudio Africa,
    Daniel Arndt,
    Wolfgang Bangerth,
    Bruno Blais,
    Marc Fehling,
    Rene Gassm\"{o}ller,
    Timo Heister,
    Luca Heltai,
    Martin Kronbichler,
    Matthias Maier,
    Peter Munch,
    Magdalena Schreter-Fleischhacker,
    Jan Philipp Thiele,
    Bruno Turcksin,
    David Wells,
    Vladimir Yushutin
  },
  pdftitle={The deal.II Library, Version 9.6, 2024},
}

\title{The \dealii Library, Version 9.6}

 \author[1*]{Pasquale~Claudio~Africa}
 \affil[1]{SISSA International School for Advanced Studies,
   mathLab,
   Via Bonomea, 265,
   34136, Trieste, Italy.
   \texttt{pafrica@sissa.it}}

 \author[2*]{Daniel Arndt}
 \affil[2]{Computational Coupled Physics Group,
   Computational Sciences and Engineering Division,
   Oak Ridge National Laboratory, 1 Bethel Valley Rd.,
   TN 37831, USA.
   \texttt{arndtd/turcksinbr@ornl.gov}}

 \author[3,4]{Wolfgang~Bangerth}
 \affil[3]{Department of Mathematics, Colorado State University, Fort
   Collins, CO 80523-1874, USA.
   \texttt{bangerth@colostate.edu}}
 \affil[4]{Department of Geosciences, Colorado State University, Fort
   Collins, CO 80523, USA.}

\author[5]{Bruno Blais}
\affil[5]{Chemical Engineering High-performance Analysis, Optimization and Simulation (CHAOS) laboratory, Department of Chemical Engineering,
             Polytechnique Montréal,
             PO Box 6079, Stn Centre-Ville, Montréal, Québec, Canada, H3C 3A7.
             {\texttt{bruno.blais@polymtl.ca}}}

\author[6]{Marc~Fehling}
 \affil[6]{Department of Mathematical Analysis,
    Faculty of Mathematics and Physics, Charles University,
    Sokolovsk{\'a} 49/83, 186\,75 Prague 8, Czech Republic.
    {\texttt{marc.fehling@matfyz.cuni.cz}}}


\author[7]{Rene~Gassm\"{o}ller}
\affil[7]{GEOMAR Helmholtz Centre for Ocean Research Kiel, 24148 Kiel, Germany}

\author[8]{Timo~Heister}
 \affil[8]{School of Mathematical and Statistical Sciences,
   Clemson University,
   Clemson, SC, 29634, USA.
   {\texttt{heister@clemson.edu}}}

\author[9]{Luca~Heltai}
\affil[9]{University of Pisa, Italy.}

\author[10]{Sebastian~Kinnewig}
\affil[10]{Institute for Applied Mathematics, Scientific Computing, 
           Leibniz University Hannover, 
           Welfengarten 1, 30167 Hannover, Germany
           {\texttt{kinnewig@ifam.uni-hannover.de}}}

 \author[11,12]{Martin~Kronbichler}
 \affil[11]{Faculty of Mathematics, Ruhr University Bochum,
   Universit\"atsstr.~150, 44780 Bochum, Germany.
 {\texttt{martin.kronbichler@rub.de}}}
 \affil[12]{Institute of Mathematics,
   University of Augsburg,
   Universit\"atsstr.~12a, 86159 Augsburg, Germany.
   }

\author[13]{Matthias~Maier}
\affil[13]{Department of Mathematics,
  Texas A\&M University,
  3368 TAMU,
  College Station, TX 77845, USA.
  {\texttt{maier@math.tamu.edu}}}

\author[12,14]{Peter Munch}
 \affil[14]{Uppsala University, Sweden.
  {\texttt{peter.munch@it.uu.se}}}

\author[15]{Magdalena Schreter-Fleischhacker}
 \affil[15]{Institute for Computational Mechanics, Technical University of Munich, Boltzmannstraße 15, 85748 Garching, Germany.
  {\texttt{magdalena.schreter@tum.de}}}

\author[16]{Jan Philipp Thiele}
 \affil[16]{Weierstrass Institute for Applied Analysis and Stochastics,\newline
 Leibniz Institute in Forschungsverbund Berlin e.V.
  {\texttt{thiele@wias-berlin.de}}}

\author[2]{Bruno~Turcksin}

\author[17]{David Wells}
\affil[17]{Department of Mathematics, University of North Carolina,
  Chapel Hill, NC 27516, USA.
  {\texttt{drwells@email.unc.edu}}}

\author[18,8]{Vladimir Yushutin}
\affil[18]{Department of Mathematics, University of Tennessee at Knoxville,
	 Knoxville TN 37996-1320, USA.
{\texttt{vyushuti@utk.edu}}}


\renewcommand{\labelitemi}{--}


\begin{document}
\maketitle

\footnotetext{%
  $^\ast$ This manuscript has been authored by UT-Battelle, LLC under Contract No.
  DE-AC05-00OR22725 with the U.S. Department of Energy.
  % We need to submit the manuscript with the text below. If the editor
  % complains we can remove it.
  The United States
  Government retains and the publisher, by accepting the article for
  publication, acknowledges that the United States Government retains a
  non-exclusive, paid-up, irrevocable, worldwide license to publish or reproduce
  the published form of this manuscript, or allow others to do so, for United
  States Government purposes. The Department of Energy will provide public
  access to these results of federally sponsored research in accordance with the
  DOE Public Access Plan (http://energy.gov/downloads/doe-public-access-plan).
}


\begin{abstract}
  This paper provides an overview of the new features of the finite element
  library \dealii, version 9.6.
\end{abstract}



%%%%%%%%%%%%%%%%%%%%%%%%%%%%%%%%%%%%%%%%%%%%%%%%%%%%%%%%%%%%%%%%%%%%%%%%%%%%%%%%
%%%%%%%%%%%%%%%%%%%%%%%%%%%%%%%%%%%%%%%%%%%%%%%%%%%%%%%%%%%%%%%%%%%%%%%%%%%%%%%%
%%%%%%%%%%%%%%%%%%%%%%%%%%%%%%%%%%%%%%%%%%%%%%%%%%%%%%%%%%%%%%%%%%%%%%%%%%%%%%%%
\section{Overview}

\dealii version 9.6.0 was released August 11, 2024.
This paper provides an
overview of the new features of this release and serves as a citable
reference for the \dealii software library version 9.6. \dealii is an
object-oriented finite element library used around the world in the
development of finite element solvers. It is available for free under the
GNU Lesser General Public License (LGPL). Downloads are available at
\url{https://www.dealii.org/} and \url{https://github.com/dealii/dealii}.

The major changes of this release are:
%
\begin{itemize}
  \item Substantial performance improvements to the matrix-free and multigrid
   infrastructure (Section~\ref{sec:mf}).
  \item Different additions to the non-matching infrastructure (see
    Section~\ref{sec:nonmatching}). In particular, the new class
    FERemoteEvaluation provides support for evaluating
    finite element shape functions and solutions on parts of the
    domain stored by other MPI processes.
  \item Much work has gone into writing wrappers for Trilinos' Tpetra
    stack of linear algebra classes. Tpetra is Trilinos' Kokkos-based
    replacement for the now-deprecated Epetra stack. See
    Section~\ref{sec:tpetra} for more on this.
  \item Tool classes \texttt{TaskResult} and \texttt{Lazy} that provide ways to compute
    values on a separate thread or only when first accessed (Section~\ref{sec:tools}).
  \item There are six new tutorial programs, on checkpointing simulations
    (step-83), integrating time-dependent solvers with external time
    stepping libraries (step-86, using PETSc's TS library), advanced
    point evaluation techniques (step-87), non-matching grids
    (step-89), and trace-based methods for PDEs on embedded surfaces
    (step-90). See Section~\ref{subsec:steps} for more details.
\end{itemize}
%

\todo[inline]{If we keep the section on relicensing
  (Section~\ref{sec:license}), mention it in the list above.}

While all of these major changes are discussed in detail in
Section~\ref{sec:major}, there
are a number of other noteworthy changes in the current \dealii release,
which we briefly outline in the remainder of this section:
%
\begin{itemize}
  \item deal.II now requires and uses of C++17.
  \item We have continued to make progress in supporting simplex and
    mixed meshes -- mesh types that \dealii{} has traditionally not
    supported at all. Specifically, the current release uses
    better strategies for refinement of tetrahedra that result in
    better-shaped child cells. It also contains
    support for cubic finite elements on simplices.
  \item The \texttt{FE\_NedelecSZ} class that contains our
    implementation of the N\'ed\'elec element using the orientation
    scheme of \cite{Zag06} now supports the computation of hanging
    node constraints for locally refined meshes.
    \todo[inline]{Sebastian: Is there anything more to say here, maybe some reference? If
      you want, we can also list this point in the list above, and add
      a whole section 2.x for it.}
  \item changes of interface of AffineConstraints
    \todo[inline]{Wolfgang to write}
  \item \dealii{} internally needs to keep track of the relative
    orientations of the coordinate systems associated with neighboring
    cells, as well as of the coordinate systems of faces and edges
    within the coordinate system of the cells they are part
    of. Historically, different places within the library that had to
    do that had grown their own, independent ways of representating
    this information, much of which was difficult to understand.
    We have now cleaned up this area. Specifically, instead of using three
    Booleans to define relative orientations for hexedra (\textit{orientation},
    \textit{flip}, and \textit{rotation}), we have internally merged these three values into a single variable of type
     \texttt{char} -- i.e., the smallest integer data type that has at
     least three bits to represent the information. Because the way to represent
     orientations is part of the public interface, the change causes
     some incompatibilities in types and function signatures.
  \todo[inline]{DavidW: Please describe some details here. -- When
    done, please remove this comment.}
\item The \texttt{SolverGMRES} class now offers a third orthogonalization
  method, the classical Gram--Schmidt method with delayed orthogonalization
  \cite{Bielich2022}. Furthermore, the solver now specifies the maximal basis
  size of the Arnoldi basis, rather than the number of auxiliary
  vectors. Some changes have also been made to the GMRES and F-GMRES
  implementations, making them use the same underlying kernels as much as possible.
\end{itemize}
%
The changelog lists more than 180 other features and bugfixes.


%%%%%%%%%%%%%%%%%%%%%%%%%%%%%%%%%%%%%%%%%%%%%%%%%%%%%%%%%%%%%%%%%%%%%%%%%%%%%%%%
%%%%%%%%%%%%%%%%%%%%%%%%%%%%%%%%%%%%%%%%%%%%%%%%%%%%%%%%%%%%%%%%%%%%%%%%%%%%%%%%
%%%%%%%%%%%%%%%%%%%%%%%%%%%%%%%%%%%%%%%%%%%%%%%%%%%%%%%%%%%%%%%%%%%%%%%%%%%%%%%%
\section{Major changes to the library}
\label{sec:major}

This release of \dealii contains a number of large and significant changes,
which we will discuss in this section.
It of course also includes a
vast number of smaller changes and added functionality; the details of these
can be found
\href{https://dealii.org/developer/doxygen/deal.II/changes_between_9_5_2_and_9_6_0.html}
{in the file that lists all changes for this release}; see \cite{changes96}.



%%%%%%%%%%%%%%%%%%%%%%%%%%%%%%%%%%%%%%%%%%%%%%%%%%%%%%%%%%%%%%%%%%%%%%%%%%%%%%%%
\subsection{Updates to multigrid and matrix-free algorithms}\label{sec:mf}

%\todo[inline]{Martin/Peter: The list in the introduction also mentions the two-level
%  operators. Is this the right place to also discuss these?}
  
We made different updates to the matrix-free infrastructure in \dealii. These include:
\begin{itemize}
\item Our own implementation of \texttt{std::simd} (\texttt{VectorizedArray}) now
  also supports Arm Neon. Arm Neon is an architecture extension for
  the Arm Cortex-A and Arm Cortex-R series of processors, commonly used in Apple
  products. With these instructions 2 doubles or 4 floats can be processed in one
  go. Since matrix-free infrastructure works directly with \texttt{VectorizedArray}
  data structures, it automatically benefits from this implementation extension.
  
\item We made some improvement for evaluation of values and gradients for
  $H$(div)-conforming Raviart--Thomas elements on non-Cartesian elements.
  
\item We furthermore improved the internal data structures of the
  tensor-product evaluators as well as the evaluators for simplex elements,
  which speed up the evaluation in several scenarios, especially multi-component systems.
\end{itemize}

Furthermore, we performed substantial improvements to the global-coarsening
multigrid infrastructure: \texttt{MGTransferMF} (previously:
\texttt{MGTransferGlobalCoarsening}) and \texttt{MGTwoLevelTransfer}. They
now allow to perform local smoothing, which is a key step towards unifying
all transfer operators in \dealii. \texttt{MGTwoLevelTransfer} now also
can be set up with an existing \texttt{MatrixFree} object in the case
of $p$-multigrid, allowing to reduce the setup costs significantly.


%%%%%%%%%%%%%%%%%%%%%%%%%%%%%%%%%%%%%%%%%%%%%%%%%%%%%%%%%%%%%%%%%%%%%%%%%%%%%%%%
\subsection{Advances in non-matching support}\label{sec:nonmatching}

%\todo[inline]{All: If you contributed to this functionality, please
%  help flesh out this section.}
  
We made different advances to the non-matching infrastructure in \dealii. These include:

\begin{itemize}
\item The performance of the non-nested multigrid infrastructure
(\texttt{MGTwoLevelTransferNon\-Nested}) has been improved significantly by avoiding
redundant copy operations. Furthermore, the support for simplex-shaped cells and
multiple-component elements have been added.
\item The new class \texttt{FERemoteEvaluation} has been added. 
This is a class to access data
in a distributed matrix-free loop for non-matching discretizations.
Interfaces are named with \texttt{FEEvaluation} in mind. Key component is the
underlying MPI communication, which is done via \texttt{RemotePointEvaluation}.
Tutorial \texttt{step-89} has been added to present its usage
in the context of an application to acoustic conservation equations~\cite{heinz2023high}.
\item FECouplingValues

\todo[inline]{Luca: Please complete}
\end{itemize}

%%%%%%%%%%%%%%%%%%%%%%%%%%%%%%%%%%%%%%%%%%%%%%%%%%%%%%%%%%%%%%%%%%%%%%%%%%%%%%%%
\subsection{Interface to the \trilinos{} \tpetra stack}\label{sec:tpetra}

\dealii{}'s parallel linear algebra facilities are largely built on
wrappers around functionality provided by the \petsc{} and \trilinos{} libraries.
As a software project, \trilinos{} is designed as a stack of semi-independent
packages building upon each other and working together. 
As a foundation, \teuchos{} offers a set of basic tools,
including parameter handling through XML files or command line interfaces 
and the reference counted pointer \texttt{RCP} which predates 
the smart pointer types introduced in \texttt{C++11}.
The second layer in the stack is a package providing distributed linear algebra 
classes for vectors and (sparse) matrices based on the \petra{}
object model. The third layer is a set of interconnected packages
providing preconditioners, various types of solvers 
as well as more advanced techniques like automatic differentiation, 
time integration and discretization.

Historically, \trilinos{} has implemented \petra{} objects -- such as
parallel vector, sparsity pattern, and sparse matrix classes -- in its
\epetra{} package that uses \MPI{} as its only source of
parallelism. The \dealii{} interfaces to \trilinos{} are therefore all
built using \epetra{}.
However, several years ago, \trilinos also introduced the
newer \tpetra{} (``templated'' Petra) package that provides additional shared memory 
parallelism and GPU capabilities by building on \kokkos{}. \tpetra{}
is slated to eventually replace \epetra{}, and the latter is indeed
now deprecated with removal slated for 2025. As a consequence, we will
eventually have to switch all of our \trilinos{} interfaces to
\tpetra{}; the same is true for the need to switch to \tpetra{}-based
sub-packages replacing existing \trilinos sub-packages (for example,
Ifpack2 instead of Ifpack).

In the current release, we have put substantial work into this switch.

\todo[inline]{Sebastian \& Jan Philipp: Please complete}


\begin{itemize}
  \item Short overview capabilities of \epetra (MPI parallel, fixed types) [Thiele]
  \item Short overview capabilities of \tpetra (MPI+X via Kokkos, auto diff types, Templated petra) [Thiele]
    \todo[inline]{Keep discussion of the two items above as short as
      you can -- we're describing the changes in deal.II, not some
      other package. Perhaps roll what you want to say into the
      paragraphs above?}
  \item Current state of the interface (Tpetra based linear algebra, direct solver from Amesos2, preconditioners from Ifpack2) [Thiele \& Kinnewig]
  \item User relevant changes from \epetra: incompatibilities, removed
    options, changes in the constructor. [Thiele \& Kinnewig]
    \todo[inline]{These last two items are the important ones.}
\end{itemize}

%%%%%%%%%%%%%%%%%%%%%%%%%%%%%%%%%%%%%%%%%%%%%%%%%%%%%%%%%%%%%%%%%%%%%%%%%%%%%%%%
\subsection{More support for advanced programming idioms}\label{sec:tools}

\todo[inline]{Wolfgang to write. Mention Lazy and TaskResult.}


%%%%%%%%%%%%%%%%%%%%%%%%%%%%%%%%%%%%%%%%%%%%%%%%%%%%%%%%%%%%%%%%%%%%%%%%%%%%%%%%
\subsection{New and improved tutorials and code gallery programs}
\label{subsec:steps}

Many of the \dealii tutorial programs were revised in a variety of ways
as part of this release. In addition, there are a number of new tutorial
programs:
\begin{itemize}
  \item
    \texttt{step-83} 
    demonstrates how one can implement
    checkpoint/restart functionality in \dealii{}-based programs,
    using the BOOST serialization functionality as a
    foundation. step-83 was written by Pasquale Africa, Wolfgang
    Bangerth, and Bruno Blais and uses step-19 as its basis.
  \item
    \texttt{step-86}
    is a program that solves the heat equation using PETSc's TS (time
    stepping) framework for the solution of ordinary differential
    equations. Written by Wolfgang Bangerth (Colorado State
    University), Luca Heltai (University of Pisa), and Stefano Zampini
    (King Abdullah University of Science and Technology), it
    illustrates how PDE solvers for time-dependent problems can be
    integrated with existing ODE solver packages to use advanced ODE
    solver concepts (such as higher-order time integration methods and
    adaptive time step control), all without sacrificing the things
    that have traditionally led code authors toward writing their own
    time stepping routines (such as wanting to change the mesh every
    once in a while, or having to deal with boundary conditions).
  \item
    \texttt{step-87} was contributed by Magdalena Schreter-Fleischhacker
    (Technical University of Munich) and Peter Munch
    (University of Augsburg/Uppsala University). It
    presents the advanced point-evaluation functionalities of \dealii
    that are useful for evaluating finite element solutions at
    arbitrary points on finite element meshes that can be distributed among processes.
  \item
    \texttt{step-89} was contributed by Johannes Heinz (TU Wien),
    Maximilian Bergbauer (Technical University of Munich),
    Marco Feder (SISSA), and Peter Munch (University of Augsburg/Uppsala University).
    It shows one way how to apply non-matching and/or Chimera methods
    within matrix-free loops in \dealii.
  \item
    \texttt{step-90} was contributed by Vladimir Yushutin and Timo Heister (Clemson University).
    It implements the trace finite element method (TraceFEM). TraceFEM solves PDEs
    posed on a, possibly evolving, $(dim-1)$-dimensional surface $\Gamma$ employing
    a fixed uniform background mesh of a $dim$-dimensional domain in which
    the surface is embedded. Such surface PDEs arise in problems involving
    material films with complex properties and in other situations in which
    a non-trivial condition is imposed on either a stationary or a moving interface.
    The program considers a steady, complex, non-trivial surface and the prototypical
    Laplace-Beltrami equation which is a counterpart of
    the Poisson problem on flat domains.
\end{itemize}

In addition, there are three new programs in the code gallery (a collection of
user-contributed programs that often solve more complicated problems
than tutorial programs, and that are intended as starting points for further
research rather than as teaching tools):
UPDATE
\begin{itemize}
  \item \textit{``Crystal growth phase field model'''},
    contributed by Umair Hussain;
  \item \textit{``Nonlinear heat transfer problem''}, contributed by
    Narasimhan Swaminathan;
  \item \textit{``Traveling-wave solutions of a qualitative model for combustion waves''}, contributed by
    Shamil Magomedov.
\end{itemize}

Furthermore, we added an example to the \texttt{libCEED}
library.\footnote{\url{https://github.com/CEED/libCEED}} \texttt{libCEED} is a library
that provides matrix-free evaluation routines that work on different hardware. The
examples show how to interface the \dealii data structures to the \texttt{libCEED} ones.
The example solves the BP1-BP6 benchmarks (scalar/vector Laplace/mass matrix with
regular integration and over-integration).


%%%%%%%%%%%%%%%%%%%%%%%%%%%%%%%%%%%%%%%%%%%%%%%%%%%%%%%%%%%%%%%%%%%%%%%%%%%%%%%%
\subsection{Relicensing to Apache 2.0}\label{sec:license}

\todo[inline]{Matthias: Where are we with this, and what of it is
  worth describing here?}



%%%%%%%%%%%%%%%%%%%%%%%%%%%%%%%%%%%%%%%%%%%%%%%%%%%%%%%%%%%%%%%%%%%%%%%%%%%%%%%%
\subsection{Incompatible changes}\label{subsec:deprecated}

The 9.6 release includes
\href{https://dealii.org/developer/doxygen/deal.II/changes_between_9_5_2_and_9_6_0.html}
     {around 40 incompatible changes};
see \cite{changes96}. Many of these
incompatibilities change internal
interfaces that are not usually used in external
applications. That said, the following are worth mentioning since they
may have been more widely used:
\begin{itemize}
  \item \dealii{} now requires compilers to support C++17, and has
    started to extensively use C++17 features.
  \item The \texttt{CUDAWrappers} namespace and its contents --
    notably things that enable the usage of cuSPARSE algorithms --
    have been deprecated and will be removed in the next
    release. Kokkos is now used for device-specific optimizations.
\end{itemize}



%%%%%%%%%%%%%%%%%%%%%%%%%%%%%%%%%%%%%%%%%%%%%%%%%%%%%%%%%%%%%%%%%%%%%%%%%%%%%%%%
%%%%%%%%%%%%%%%%%%%%%%%%%%%%%%%%%%%%%%%%%%%%%%%%%%%%%%%%%%%%%%%%%%%%%%%%%%%%%%%%
%%%%%%%%%%%%%%%%%%%%%%%%%%%%%%%%%%%%%%%%%%%%%%%%%%%%%%%%%%%%%%%%%%%%%%%%%%%%%%%%
\section{How to cite \dealii}\label{sec:cite}

In order to justify the work the developers of \dealii put into this
software, we ask that papers using the library reference one of the
\dealii papers. This helps us justify the effort we put into this library.

There are various ways to reference \dealii. To acknowledge the use of
the current version of the library, \textbf{please reference the present
  document}. For up-to-date information and a bibtex entry
see
\begin{center}
  \url{https://www.dealii.org/publications.html}
\end{center}

The original \dealii paper containing an overview of its
architecture is \cite{BangerthHartmannKanschat2007}, and a more recent
publication documenting \dealii's design decisions is available as \cite{dealII2020design}. If you rely on
specific features of the library, please consider citing any of the
following:
\begin{multicols}{2}
  \vspace*{-36pt}
  \begin{itemize}[leftmargin=4mm]
    \item For geometric multigrid: \cite{Kanschat2004,JanssenKanschat2011,ClevengerHeisterKanschatKronbichler2019, munch2022gc};
    \item For distributed parallel computing: \cite{BangerthBursteddeHeisterKronbichler11};
    \item For $hp$-adaptivity: \cite{BangerthKayserHerold2007,fehlingbangerth2023};
    \item For partition-of-unity (PUM) and finite element enrichment methods:
           \cite{Davydov2016};
    \item For matrix-free and fast assembly techniques:
          \cite{KronbichlerKormann2012,KronbichlerKormann2019};
    \item For computations on lower-dimensional manifolds:
          \cite{DeSimoneHeltaiManigrasso2009};
    \item For curved geometry representations and manifolds:
          \cite{HeltaiBangerthKronbichlerMola2019};
    \item For integration with CAD files and tools:
          \cite{HeltaiMola2015};
    \item For boundary element computations:
          \cite{GiulianiMolaHeltai-2018-a};
    \item For the \texttt{LinearOperator} and
      \texttt{Packaged\-Operation} facilities:
          \cite{MaierBardelloniHeltai-2016-a,MaierBardelloniHeltai-2016-b};
    \item For uses of the \texttt{WorkStream} interface:
          \cite{TKB16};
    \item For uses of the \texttt{ParameterAcceptor} concept, the
          \texttt{MeshWorker::ScratchData} base class, and the
          \texttt{ParsedConvergenceTable} class:
          \cite{SartoriGiulianiBardelloni-2018-a};
    \item For uses of the particle functionality in \dealii:
          \cite{GLHPB18}.
          \vfill\null
  \end{itemize}
\end{multicols}

\dealii can interface with many other libraries:
\begin{multicols}{3}
  \begin{itemize}[leftmargin=4mm]
    \item ADOL-C \cite{griewank1996adolc}
    \item ArborX \cite{lebrun2020arborx}
    \item ARPACK \cite{lehoucq1998arpack}
    \item Assimp \cite{schulze2021assimp}
    \item BLAS and LAPACK \cite{anderson1999lapack}
    \item Boost \cite{boost-web-page}
    \item CGAL \cite{cgal-user-ref}
    \item cuSOLVER \cite{cusolver-web-page}
    \item cuSPARSE \cite{cusparse-web-page}
    \item Gmsh \cite{geuzaine2009gmsh}
    \item GSL \cite{galassi2009gsl,gsl-web-page}
    \item Ginkgo \cite{anzt2020ginkgo,anzt2022ginkgo}
    \item HDF5 \cite{hdf5-web-page}
    \item METIS \cite{karypis1998metis}
    \item MUMPS \cite{amestoy2001mumps,amestoy2019mumps}
    \item muparser \cite{muparser-web-page}
    \item OpenCASCADE \cite{opencascade-web-page}
    \item p4est \cite{burstedde2011p4est}
    \item PETSc \cite{petsc-user-ref,petsc-web-page}
    \item ROL \cite{ridzal2014rol}
    \item ScaLAPACK \cite{blackford1997scalapack}
    \item SLEPc \cite{hernandez2005slepc}
    \item SUNDIALS \cite{hindmarsh2005sundials}
    \item SymEngine \cite{symengine-web-page}
    \item Taskflow \cite{huang2021taskflow}
    \item TBB \cite{reinders2007tbb}
    \item Trilinos \cite{heroux2005trilinos,trilinos-web-page}
    \item UMFPACK \cite{davis2004umfpack}
  \end{itemize}
\end{multicols}
Please consider citing the appropriate references if you use
interfaces to these libraries.

The two previous releases of \dealii can be cited as
\cite{dealII92,dealII93}.


\section{Acknowledgments}

\dealii is a worldwide project with dozens of contributors around the
globe. Other than the authors of this paper, the following people
contributed code to this release:\\

% This is up-to-date as of 2024-07-23 11:52PM CEST
\begin{quote}
Laryssa      Abdala,
Mathias      Anselmann,
Abbas        Ballout,
Maximilian   Bergbauer,
Julian       Brotz,
Marco        Feder,
Niklas       Fehn,
Menno        Fraters,
Quang        Hoang,
Vladimir     Ivannikov,
Tao          Jin,
Yimin        Jin,
Sebastian    Kinnewig,
Paras        Kumar,
Sébastien    Loriot,
Nils         Much,
Abdullah     Mujahid,
Bob          Myhill,
Paul A.      Patience,
Luz          Paz,
Laura        Prieto Saavedra,
Sebastian    Proell,
Hendrik      Ranocha,
Johannes     Resch,
Andreas      Ritthaler,
Malik        Scheifinger,
David        Schneider,
%Magdalena    Schreter-Fleischhacker,
Richard      Schussnig,
Nils         Schween,
Kyle         Schwiebert,
Simranjeet   Singh,
Simon        Sticko,
Dominik      Still,
%Jan Philipp  Thiele,
Thierry      Thomas,
Vinayak      Vijay,
Ivy          Weber,
Simon        Wiesheier,
Chengjiang   Yin,
%Vladimir     Yushutin,
Stefano      Zampini.
\end{quote}
Their contributions are much appreciated!


\bigskip

\dealii and its developers are financially supported through a
variety of funding sources:

P.~C.~Africa was partially supported by the consortium iNEST (Interconnected North-East Innovation Ecosystem),
Piano Nazionale di Ripresa e Resilienza (PNRR) - Missione 4 Componente 2, Investimento 1.5 - D.D. 1058 23/06/2022,
ECS00000043, supported by the European Union's NextGenerationEU program.

D.~Arndt and B.~Turcksin: Research sponsored by the Laboratory Directed Research and
Development Program of Oak Ridge National Laboratory, managed by UT-Battelle,
LLC, for the U. S. Department of Energy.

W.~Bangerth and M.~Fehling were partially supported by Award OAC-1835673
as part of the Cyberinfrastructure for Sustained Scientific Innovation (CSSI)
program.

W.~Bangerth, T.~Heister, and R.~Gassm\"{o}ller were partially
supported by the Computational Infrastructure for Geodynamics initiative
(CIG), through the National Science Foundation (NSF) under Award
No.~EAR-2149126 via The University of California -- Davis.

W.~Bangerth was also partially supported by Award EAR-1925595.

B.~Blais was supported by the National Science and Engineering Research Council of Canada (NSERC)  through the RGPIN-2020-04510 Discovery Grant and the MMIAOW Canada Research Level 2 in Computer-Assisted Design and Scale-up of Alternative Energy Vectors for Sustainable Chemical Processes.

M.~Fehling was also partially supported by the ERC-CZ grant LL2105
CONTACT, funded by the Czech Ministry of Education, Youth and Sports.

R.~Gassm\"{o}ller was also partially supported by NSF Awards EAR-1925677
and EAR-2054605.

T.~Heister and V.~Yushutin were also partiallty supported by NSF Awards OAC-2015848 and EAR-1925575.

T.~Heister was also partially supported by NSF OAC-2410848.

L.~Heltai was partially supported by the Italian Ministry of
University and Research (MUR), under the grant MUR PRIN 2022 No. 2022WKWZA8
``Immersed methods for multiscale and multiphysics problems (IMMEDIATE)''.

\todo[inline]{Sebastian: Anything to list here?}

M.~Kronbichler and P.~Munch were partially supported by the
German Ministry of Education and Research, project
``PDExa: Optimized software methods for solving partial differential
equations on exascale supercomputers'' and the Bayerisches Kompetenznetzwerk
f\"ur Technisch-Wissen\-schaft\-li\-ches Hoch- und H\"ochstleistungsrechnen
(KONWIHR), projects ``High-order matrix-free finite
element implementations with hybrid parallelization and improved data
locality'' and ``Fast and scalable finite element algorithms for coupled
multiphysics problems and non-matching grids''.

M.~Maier was partially supported by NSF Award DMS-2045636 and and by the
Air Force Office of Scientific Research under grant/contract number
FA9550-23-1-0007.

\todo[inline]{Peter: Anything to list here?}

M.~Schreter-Fleischhacker was supported by the Austrian Science Fund (FWF) Schrödinger Fellowship (project number: J4577) and by the European Research Council
through the ERC Starting Grant ExcelAM (project number: 101117579).

\todo[inline]{Jan Philipp: Anything to list here?}

D.~Wells was supported by NSF Award OAC-1931516.

Clemson University is acknowledged for generous allotment of compute
time on the Palmetto cluster.

The authors acknowledge the Texas Advanced Computing Center (TACC) at The University of Texas at Austin for providing HPC resources that have contributed to the research results reported within this paper. \url{http://www.tacc.utexas.edu}

\todo[inline]{Marc: please insert allocation}

This work used [resource-name] at [resource provider] through allocation [allocation number] from the Advanced Cyberinfrastructure Coordination Ecosystem: Services \& Support (ACCESS) program, which is supported by National Science Foundation grants \#2138259, \#2138286, \#2138307, \#2137603, and \#2138296. See \cite{Boerner2023}.


This work used the Extreme Science and Engineering Discovery Environment (XSEDE), which is supported by National Science Foundation grant number ACI-1053575 access through the CIG Science Gateway and Community Codes for the Geodynamics Community MCA08X011 allocation.


\bibliography{paper}{}
\bibliographystyle{abbrv}

\end{document}
